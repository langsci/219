\chapter[]{Nominal forms in discourse: the alternation of third-person singular pronouns}\label{alternation}
\lehead{Nominal forms in discourse: the alternation of 3rd-person singular pronouns}
As mentioned previously in \sectref{izpronouns}, \tabref{izpronounstable}, third-person dependent and independent pronouns both alternate between an overt form (\textit{=be}) and a \isi{zero form} (\textit{=$\varnothing$}). Because the choice between the overt and the \isi{zero form} is free at the main clause level in both transitive and intransitive constructions, an explanation of the differential distribution between the two requires a more detailed syntactic and pragmatic analysis. This is the subject of this section, which begins with a discussion of the syntactic facts constraining the distribution of either pronominal form and then moves to an analysis of the discursive motivations involved in their use. In order to offer a more complete view, in addition to the Pear Story corpus, the analysis here also draws from previously published studies, from data collected using elicitation techniques, and from spontaneous dialogue. 


\section{Syntactic constraints on the overt versus zero alternation}

The \isi{zero form} has a more constrained syntactic distribution than the overt form, that is, the \isi{zero form} has a narrower set of binding conditions. This can be observed in the case of reflexives and dependent clauses. 


\subsection{Reflexives}

The reflexive consists of the word \textit{laaca} `same' followed by an \isi{independent pronoun} co-indexed with its antecedent. The zero pronoun is bound by a full NP antecedent (\ref{reflexive3}) or another zero pronoun (\ref{reflexive2}): 

\ea\label{reflexive3}
\glll biiya B\v{e}tu\textsubscript{1} laaca l\'{a}a\textsubscript{1} \\
bi=uuya Be\textsuperscript{LH}tu laaca\textsuperscript{LH} laa={$\varnothing$} \\
\textsc{compl}=see Betu \textsc{same} \textsc{base}=\textsc{3} \\
\glt `Betu saw himself.' 
\z

\ea\label{reflexive2} 
\glll biiya\textsubscript{1} laaca l\'{a}a\textsubscript{1} \\
bi=uuya={$\varnothing$} laaca\textsuperscript{LH} laa={$\varnothing$} \\
\textsc{compl}=see=\textsc{3} \textsc{same} \textsc{base}=\textsc{3} \\
\glt `S/he saw himself/herself.'
\z
Meanwhile, the overt pronoun can only be bound by another overt pronoun, as shown in (\ref{reflexive1})-(\ref{reflexive5}):

\ea\label{reflexive1}
\glll biiyabe\textsubscript{1} l\'{a}ac\'{a} l\'{a}ab\v{e}\textsubscript{1} \\
bi=uuya=be\textsuperscript{LH}  laaca\textsuperscript{LH} laa=be\textsuperscript{LH}  \\
\textsc{compl}=see=\textsc{3.hum} \textsc{same} \textsc{base}=\textsc{3.hum} \\
\glt `S/he saw himself/herself.'
\z

\ea\label{reflexive4}
\glll biiya B\v{e}tu\textsubscript{1} (*laaca) laab\v{e}\textsubscript{1} \\
bi=uuya Be\textsuperscript{LH}tu (laaca\textsuperscript{LH}) laa=be\textsuperscript{LH}  \\
\textsc{compl}=see Betu (\textsc{same}) \textsc{base}=\textsc{3.hum} \\
\glt `Betu saw him/her (*himself).'
\z

\ea\label{reflexive5}
\glll biiya\textsubscript{2} (*laaca) laab\v{e}\textsubscript{1} \\
bi=uuya={$\varnothing$} (laaca\textsuperscript{LH}) laa=be\textsuperscript{LH}  \\
\textsc{compl}=see=\textsc{3} (\textsc{same}) \textsc{base}=\textsc{3.hum} \\
\glt `S/he saw him/her (*himself).'
\z
Therefore, the overt form can only co-refer with another overt form and a \isi{zero form} can co-refer with either a full NP or a \isi{zero form}, but not an overt form, within the main clause. A similar situation holds for dependent clauses.


\subsection{Dependent clauses}

An overt third-person pronominal subject in a dependent clause cannot co-refer to the subject NP in the main clause:

\ea\label{control6}
\glll racaladxi B\v{e}tu\textsubscript{2} gu\'{e}ed\'{a}b\'{e}\textsubscript{1} \'{i}x\'{i}' \\
ri=aca-ladxi Be\textsuperscript{LH}tu gu\textsuperscript{LH}=eeda\textsuperscript{LH}=be\textsuperscript{LH}  guixi'\textsuperscript{H}  \\
\textsc{hab}=occur-gut Betu \textsc{pot}=come=\textsc{3.hum} tomorrow \\
\glt `Betu wants him/her to come tomorrow.'  \hfill{(MP 13)}\footnote{If the example is not from my own corpus, I refer to the source of the examples using the following notation: 
MP= \citet{marlett1996};
PBC= \citet{pickett1998}; 
M= \citet{marlett1993}.
The number that follows refers to the example number in the source.}
\z
The overt form in the dependent clause cannot refer to Betu. Instead, a \isi{zero form} must be used (\ref{control5}):

\ea\label{control5}
\glll racaladxi B\v{e}tu\textsubscript{1} gu\'{e}ed\'{a}\textsubscript{1} \'{i}x\'{i}' \\
ri=aca-ladxi Betu gu\textsuperscript{LH}=eeda\textsuperscript{LH}={$\varnothing$} guixi'\textsuperscript{H}  \\
\textsc{hab}=occur-gut Betu \textsc{pot}=come=\textsc{3} tomorrow \\
\glt `Betu wants to come tomorrow.' \hfill{(MP 22)}
\z
Identical pronominal forms obligatorily co-refer across dependent clauses, as in (\ref{control1}), (\ref{control2}):

\ea\label{control1}
\glll racaladxibe\textsubscript{1} gu\'{e}ed\'{a}b\'{e}\textsubscript{1} \'{i}x\'{i}' \\
ri=aca-ladxi=be\textsuperscript{LH}  gu\textsuperscript{LH}=eeda\textsuperscript{LH}=be\textsuperscript{LH}  guixi'\textsuperscript{H}  \\
\textsc{hab}=occur-gut=\textsc{3.hum} \textsc{pot}=come=\textsc{3.hum} tomorrow \\
\glt `S/he wants to come tomorrow.'
\z

\ea\label{control2}
\glll racaladxi\textsubscript{1} gu\'{e}ed\'{a}\textsubscript{1} \'{i}x\'{i}' \\
ri=aca-ladxi={$\varnothing$} gu=eeda\textsuperscript{LH}={$\varnothing$} guixi'\textsuperscript{H}  \\
\textsc{hab}=occur-gut=\textsc{3} \textsc{pot}=come=\textsc{3} tomorrow \\
\glt `S/he wants to come tomorrow.'
\z
They may both either be overt or both zero. In contrast, non-identical pronominal forms do not co-refer, as shown in (\ref{control3}), (\ref{control4}):

\ea\label{control3}
\glll racaladxibe\textsubscript{1} gu\'{e}ed\'{a}\textsubscript{2} \'{i}x\'{i}' \\
ri=aca-ladxi=be\textsuperscript{LH}  gu\textsuperscript{LH}=eeda\textsuperscript{LH}={$\varnothing$} guixi'\textsuperscript{H}  \\
\textsc{hab}=occur-gut=\textsc{3.hum} \textsc{pot}=come=\textsc{3} tomorrow \\
\glt `S/he wants him/her to come tomorrow.'  \hfill{(MP 34)}
\z

\ea\label{control4}
\glll racaladxi\textsubscript{2} gu\'{e}ed\'{a}b\'{e}\textsubscript{1} \'{i}x\'{i}' \\
ri=aca-ladxi={$\varnothing$} gu\textsuperscript{LH}=eeda\textsuperscript{LH}=be\textsuperscript{LH}  guixi'\textsuperscript{H}  \\
\textsc{hab}=occur-gut=\textsc{3} \textsc{pot}=come=\textsc{3.hum} tomorrow \\
\glt `S/he wants him/her to come tomorrow.' \hfill{(MP 77)}
\z
Similarly, an overt third-person pronominal object in a dependent clause cannot co-refer to a previously mentioned NP in the main clause (\ref{objectcomp1}):

\ea\label{objectcomp1}
\glll na B\v{e}tu\textsubscript{1} Y\v{e}rmo\textsubscript{2} biiya laab\v{e}\textsubscript{3} \\
na Be\textsuperscript{LH}tu Ye\textsuperscript{LH}rmo bi=uuya laa=be\textsuperscript{LH}  \\
say Betu Yermo \textsc{compl}=see \textsc{base}=\textsc{3.hum} \\
\glt `Betu$_{x}$ said Yermo$_{y}$ saw him.$_{*x, *y, z}$' \hfill{(MP 63)}
\z
The \isi{zero form} must be used for co-reference (\ref{objectcomp2})

\ea\label{objectcomp2}
\glll na B\v{e}tu\textsubscript{1} Y\v{e}rmo\textsubscript{2} biiya laa\textsubscript{1} \\
na Be\textsuperscript{LH}tu Ye\textsuperscript{LH}rmo bi=uuya laa={$\varnothing$} \\
say Betu Yermo \textsc{compl}=see \textsc{base}=\textsc{3} \\
\glt `Betu$_{x}$ said Yermo$_{y}$ saw him.$_{x, *y, *z}$' \hfill{(MP 63)}
\z
Based on evidence from reflexives and dependent clauses, then, we can say that the above generalization is true between a main clause and a dependent clause as well. That is, the overt form can only co-refer with another overt form and a \isi{zero form} can co-refer with either a full NP or a \isi{zero form}, but not an overt form.


\subsection{Adverbial clauses}

Similarly, the overt form in a pre-posed \isi{adverbial} clause cannot refer cataphorically to an NP in the main clause (\ref{adverbial1}):

\ea\label{adverbial1}
\glll \v{o}ra gu\'{e}ed\'{a}b\'{e}\textsubscript{1} l\'{a}, ze B\v{e}tu\textsubscript{2} nisa qu\v{e} \\
o\textsuperscript{LH}ra gu\textsuperscript{LH}=eeda\textsuperscript{LH}=be\textsuperscript{LH}  la\textsuperscript{H} z.e' Be\textsuperscript{LH}tu nisa que\textsuperscript{LH} \\
when \textsc{pot}=come=\textsc{3.hum} \textsc{la} \textsc{fut}.drink Betu water  \textsc{dist} \\
\glt `When he$_{*x, y}$ comes, Betu$_{x}$ will drink that water.'  \hfill{(MP 10)}
\z
Here, the use of the overt form in the \isi{adverbial} clause does not co-refer with the subject NP of the main clause. Instead, a \isi{zero form} must be used (\ref{adverbial2}):

\ea\label{adverbial2}
\glll \v{o}r\'{a} gu\'{e}ed\'{a}\textsubscript{1} l\'{a}, ze B\v{e}tu\textsubscript{1} nisa qu\v{e} \\
\v{o}ra gu=eeda\textsuperscript{LH}={$\varnothing$} la\textsuperscript{H} z.e' Be\textsuperscript{LH}tu nisa que\textsuperscript{LH} \\
when \textsc{pot}=come=\textsc{3} \textsc{la} \textsc{fut}.drink Betu water  \textsc{dist} \\
\glt `When he$_{x, *y}$ comes, Betu$_{x}$ will drink that water.' \hfill{(MP 10)}
\z
To be clear, between an \isi{adverbial} clause and a main clause, the overt form will co-refer with another overt form and a \isi{zero form} will co-refer with either a full NP or a \isi{zero form}. 

Having observed the various syntactic environments conditioning the use and co-reference of both the overt and the \isi{zero form}, the following sections explore the choices that speakers make in assigning one or other of these pronouns to referents in discourse.


\section{The overt versus zero alternation in a Pear Story monologue}

In the following excerpt from a re-telling of the Pear Story, the speaker initially assigns the overt \isi{third person form} to the man picking pears, line 04, and the \isi{zero form} to the boy on the bicycle, line 08. However, in line 14, the overt form is now used to refer to the bike boy, in the moment he rides past a new participant, the bike girl (for clarity, the overt form is marked using [\textsubscript{1}] and the \isi{zero form} using [\textsubscript{2}]):

\ea 
\begin{itemize}
\item[01]
\glll bihuiini lu ni l\'{a}, \\
bi=huiini lu ni\textsuperscript{LH} la\textsuperscript{H} \\
\textsc{compl}=appear face \textsc{3sg.inan} \textsc{la} \\
\glt `There appears,'


\item[02]
\glll ti r\'{i}gola cuchuugu caadxi cu\'{a}nanaxhi \\
ti ri\textsuperscript{H}gola c.u=chuugu' caadxi\textsuperscript{LH} cuananaxhi \\
one man \textsc{prog}.\textsc{caus}=cut few fruit \\
\glt `a man cutting some fruit.'


\item[03]
\glll r\'{i}gola que l\'{a},  \\
ri\textsuperscript{H}gola que\textsuperscript{LH} la\textsuperscript{H} \\
man \textsc{dem} \textsc{la} \\
\glt `That man,'


\item[04]
\glll m\'{a} bichabe\textsubscript{1} ch\'{u}p\'{a} dx\'{u}m\'{i} n\'{i} b\'{i}chuugub\v{e}\textsubscript{1} \\
ma'\textsuperscript{H} b.i=cha=be\textsuperscript{LH}  chupa\textsuperscript{LH} dxumi\textsuperscript{LH} ni bi=chuugu=be\textsuperscript{LH}  \\
already \textsc{compl}.\textsc{caus}=fill=\textsc{3.hum} two basket \textsc{rel} \textsc{compl}=cut=\textsc{3.hum} \\
\glt `he had already filled two baskets of pears that he cut.'


\item[05]
\glll raque c\'{u}chabe\textsubscript{1} gu\'{i}ra p\v{e}ra cuchugub\v{e}\textsubscript{1} \\
raque\textsuperscript{LH} c.u=cha=be\textsuperscript{LH}  guira\textsuperscript{LH} pe\textsuperscript{LH}ra cu-chugu=be\textsuperscript{LH}  \\
then \textsc{prog}.\textsc{caus}=put.in=\textsc{3.hum} all pear \textsc{prog}=cut=\textsc{3.hum} \\
\glt `Then he was putting in all the pears he was cutting.'


\item[06]
\glll dx\'{i}'babe\textsubscript{1} l\'{u} yaga qu\v{e} \\
dxi'\textsuperscript{H}ba=be\textsuperscript{LH}  lu yaga que\textsuperscript{LH} \\
climb=\textsc{3.hum} face tree \textsc{dist} \\
\glt `(He was) up in that tree.'


\item[07]
\glll qu\'{e} \~{n}annad\'{i}b\'{e}\textsubscript{1} b\'{e}danda t\'{i} xcu\'{i}dihuiini \\
que\textsuperscript{H} \~{n}a-nna\textsuperscript{LH}-di=be\textsuperscript{LH}  be-danda\textsuperscript{LH} ti xcui\textsuperscript{H}di-huiini \\
\textsc{neg} \textsc{irr}=know-\textsc{emph}=\textsc{3.hum} \textsc{compl}=arrive.there one boy-\textsc{dim} \\
\glt `He didn't know a boy arrived there.'
 

\item[08]
\glll dx\'{i}'ba\textsubscript{2} ti bicicl\'{e}ta  \\
dxi'\textsuperscript{H}ba={$\varnothing$} ti bicicle\textsuperscript{H}ta  \\
\textsc{part}.climb=\textsc{3} one bicycle  \\
\glt `(He was) on a bicycle.'


\item[09]
\glll gucaa\textsubscript{2} ti dxumi p\v{e}ra qu\v{e} \\
gu=caa={$\varnothing$} ti dxumi\textsuperscript{LH} pe\textsuperscript{LH}ra que\textsuperscript{LH} \\
\textsc{compl}=put=\textsc{3} one basket pear \textsc{dist} \\
\glt `(He) put that basket of pears.'


\item[10]
\glll bidx\'{i}'ba\textsubscript{2} lu xpicicl\'{e}ta\textsubscript{2} \\
bi=dxi'\textsuperscript{H}ba={$\varnothing$} lu x=bicicle\textsuperscript{H}ta={$\varnothing$}  \\
\textsc{compl}-climb=3SG face POSS=bicycle=\textsc{3} \\
\glt `(He) got on his bicycle.'

 
 \item[11]
\glll ne b\'{i}ree\textsubscript{2} ze\textsubscript{2} \\
ne\textsuperscript{LH} bi=ree={$\varnothing$} z.e={$\varnothing$} \\
and \textsc{compl}=leave=\textsc{3} \textsc{part}.go=\textsc{3} \\
\glt `And (he) left.'


\item[12]
\glll gula'na xcu\'{i}di que dx\'{u}m\'{i} p\v{e}ra stib\v{e}\textsubscript{1} \\
gu=la'na xcui\textsuperscript{H}di que\textsuperscript{LH} dxumi\textsuperscript{LH} pe\textsuperscript{LH}ra sti\textsuperscript{LH}=be\textsuperscript{LH}  \\
\textsc{compl}=steal boy \textsc{dem} basket pear \textsc{poss}=\textsc{3.hum} \\
\glt `That boy stole his basket of pears.'


\item[13]
\glll huaxa neza ze xcu\'{i}di que l\'{a}, \\
huaxa neza z.e xcui\textsuperscript{H}di que\textsuperscript{LH} la\textsuperscript{H} \\
but path \textsc{part}.go boy \textsc{dem} \textsc{la} \\
\glt `But on the path as the boy was leaving,'


\item[14]
\glll m\'{a}lasi b\'{i}dxagabe\textsubscript{1} t\'{i} badudxaapahuiini \\
ma\textsuperscript{H}lasi\textsuperscript{LH} bi=dxaga=be\textsuperscript{LH}  ti badudxaapa-huiini \\
suddenly \textsc{compl}=cross=\textsc{3.hum} one girl-\textsc{dim} \\
\glt `Suddenly he crossed a little girl'


\item[15]
\glll dx\'{i}'ba\textsubscript{2} sti bicicl\'{e}ta \\
dxi'\textsuperscript{H}ba={$\varnothing$} sti bicicle\textsuperscript{H}ta \\
\textsc{part}.climb=\textsc{3} other bicycle \\
\glt `(She was) on another bicycle.' \hfill{(\textit{Pear Stories} TVA: 4-18)\footnote{See Appendix A.}}

\end{itemize}
\z

Before line 14, the narrator refers to the bike boy using the \isi{zero form}. After line 14, the bike boy is referred to using the overt form. This switch in \isi{third person form} announces or prepares the hearer for the introduction of the girl, who is thereafter referred to using the \isi{zero form}. The bike boy, the most highly thematic participant, is referred to using the overt form for most of the remainder of the narration up until the very end, when focal attention is again paid to the pear man, who is then referred to using the overt form.

This alternating use of the overt and zero \isi{third person} forms to refer to different characters in the Pear Story is consistent across the Pear Story corpus. The pear man is consistently assigned the overt form. The bike boy is initially assigned the \isi{zero form} when he is introduced as a participant, is then assigned the overt form when the bike girl appears, and is then assigned the \isi{zero form} again when the pear man returns to the scene. The bike girl and the boy with the paddleball are consistently referred to using the \isi{zero form}. The use of the overt and zero forms across the Pear Story narratives can be summarized schematically this way:

\begin{table}

\caption{{Third person forms assigned to Pear Story referents}}
\begin{tabular}{ l c c }
\lsptoprule
& Overt form & Zero form \\

\midrule
Pear man & \checkmark & \\ 

 
Bike boy & \checkmark & \checkmark \\

 
Bike girl & & \checkmark \\

 
Boy with paddleball & & \checkmark \\

\lspbottomrule
\end{tabular}

\end{table}
Again, this pattern is consistent across all of the Pear Story narratives in the corpus. The overt form is never used with either the bike girl or the boy with the paddleball. Conversely, the \isi{zero form} is never used with the pear man. The use of the overt form coincides with the more thematic participant at each particular juncture in the narrative. This is surprising given the strong cross-linguistic tendency for highly topical participants to be zero-coded, and for overt coding to signal a change of \isi{topic} or indicate a less topical participant. In the Pear Story narratives, therefore, ZAI speakers use the distinction between the overt and zero \isi{third person} forms to assign referents varying degrees of thematicity. In the next section, I illustrate a similar use in conversation.




\section{The overt versus zero form in conversation}

In a similar way to the use in narratives described above, the overt-zero alternation can be used productively in dialogue not only to distinguish between two third-person participants but also to mutually construe one as more or less thematic than the other. The following example is taken from a conversation between two men, VA and CH. VA is asking CH about his father and goes on to ask how long each of CH's parents lived. Note, in particular, the intervention in line 06 by VA, where a zero \isi{third person form} is assigned to CH's mother (again, for clarity, the overt form is marked using [\textsubscript{1}] and the \isi{zero form} using [\textsubscript{2}]):

\ea (VA and CH, 27 Sept 2012)
\begin{itemize}
\item[01 VA:]
\glll panda \'{i}za bibani bixhozelu'? \\               
panda\textsuperscript{LH} iza bi=bani bixhoze=lu' \\
how.many year \textsc{compl}=live father=\textsc{2sg} \\
\glt `How many years did your father live?'
 

\item[02 CH:]
\glll nabanibe\textsubscript{1} c\'{e}rca de och\'{e}nta \\   
na=bani=be\textsuperscript{LH} ce\textsuperscript{H}rca de oche\textsuperscript{H}nta \\
\textsc{stat}=live=\textsc{3sg.hum} close to eighty \\
\glt `He lived close to eighty.'


\item[03 VA:]
\glll xheelabe\textsubscript{1} y\'{a}'? \\
xheela'=be\textsuperscript{LH} ya' \\
spouse=\textsc{3sg.hum} \textsc{q} \\
\glt `And his wife?'


\item[04 CH:]
\glll xheelabe\textsubscript{1} l\'{a}, \\
xheela'=be\textsuperscript{LH} la\textsuperscript{H} \\
spouse=\textsc{3sg.hum} \textsc{la}	\\
\glt `His wife,'


\item[05 CH:]
\glll laaca g\'{u}di'dibe\textsubscript{1} s\'{e}t\'{e}nta tambi\'{e}n \\
laaca\textsuperscript{LH} gu=di'di'=be\textsuperscript{LH} sete\textsuperscript{H}nta tambien\textsuperscript{H}  \\
also	COMPL-pass=3SG.HUM seventy also \\
\glt `she also passed seventy.'


\item[06 VA:]
\glll ah, laa\textsubscript{2} n\'{i}r\'{u} g\'{u}ti\textsubscript{2}  \\
ah laa={$\varnothing$} ni\textsuperscript{LH}ru\textsuperscript{LH} gu=ti={$\varnothing$}  \\
\textsc{intj} \textsc{base}=\textsc{3} front \textsc{compl}=die=\textsc{3}  \\
\glt `Ah, (she) died first.'


\item[07 CH:]
\glll prim\v{e}ru laab\v{e}\textsubscript{1}  \\                                    
prime\textsuperscript{LH}ru laa=be\textsuperscript{LH} \\
first \textsc{base}=\textsc{3sg.hum} \\
\glt `First him.'


\item[08 VA:] 	
\glll ah laabe\textsubscript{1} m\'{a}' gutib\v{e}\textsubscript{1} \\
ah laa=be\textsuperscript{LH}  ma'\textsuperscript{H} gu=ti=be\textsuperscript{LH}   \\
\textsc{intj} \textsc{base}=\textsc{3sg.hum} already \textsc{compl}=die=\textsc{3sg.hum}  \\
\glt `Ah, he already died.'


\item[09 CH:]
\glll prim\v{e}ru laab\v{e}\textsubscript{1}  \\                                    
prime\textsuperscript{LH}ru laa=be\textsuperscript{LH}   \\
first \textsc{base}=\textsc{3sg.hum}  \\
\glt `First him.'


\item[10 VA:]
\glll ah laabe\textsubscript{1} jm\'{a}ca huaniisibe\textsubscript{1} qu\'{e} j\~{n}aalu' ya'?  \\
ah laa=be\textsuperscript{LH}  jma\textsuperscript{H}ca huaniisi=be\textsuperscript{LH}  que\textsuperscript{H} j\~{n}aa=lu' ya'  \\
\textsc{intj} \textsc{base}=\textsc{3sg.hum} more old=\textsc{3sg.hum} M mother=\textsc{2sg} \textsc{q}  \\
\glt `Ah, he was older than your mother?'


\item[11 CH:]
\glll laabe\textsubscript{1} jm\'{a} huaniisib\v{e}\textsubscript{1} \\
laa=be\textsuperscript{LH}  jma\textsuperscript{H} huaniisi=be\textsuperscript{LH}   \\
\textsc{base}=\textsc{3sg.hum} more old=\textsc{3sg.hum}  \\    	 
\glt `He was older.'


\item[12 CH:]
\glll udi'dibe\textsubscript{1} l\'{u} binnig\v{o}la qu\'{e} zulu\'{a}' bia' tapa iza  \\
gu=di'di'=be\textsuperscript{LH}  lu binnigo\textsuperscript{LH}la que\textsuperscript{LH} z.ului'=a'\textsuperscript{H} bia' tapa iza  \\
\textsc{compl}=pass=\textsc{3sg.hum} face oldperson \textsc{dem} \textsc{fut}.seem=\textsc{1sg} like four year  \\
\glt `He passed the old person, I think, by about four years.'


\item[13 CH:] 
\glll peru udi'dibe\textsubscript{1} zulu\'{a}' bia' tapa iza lu j\~{n}aa'  \\
peru gu=di'di'=be\textsuperscript{LH}  z.ului'=a'\textsuperscript{H} bia' tapa iza	lu j\~{n}aa=a'\textsuperscript{H}  \\
but \textsc{compl}=pass=\textsc{3sg.hum} \textsc{fut}.seem=1SG like four year face mother=\textsc{1sg}  \\
\glt `But he passed my mother by four years.'


\item[14 CH:]
\glll jm\'{a} huaniisibe\textsubscript{1} xc\'{a}adxi  \\
jma\textsuperscript{H}	huaniisi=be\textsuperscript{LH}  xcaadxi  \\
more old=\textsc{3sg.hum} some  \\
\glt `He was a bit older.'


\item[15 VA:]
\glll {?`}dxii\~{n}a ra \~{n}aa guzaabe\textsubscript{1} d\'{e} nahuiinibe\textsubscript{1} l\'{a}?  \\
dxii\~{n}a ra \~{n}aa gu-zaa=be\textsuperscript{LH} de na-huiini=be\textsuperscript{LH} la\textsuperscript{H} \\
work \textsc{loc} field \textsc{compl}-complete=\textsc{3sg.hum} from \textsc{stat}-small=\textsc{3sg.hum} \textsc{q} \\
\glt `Did he work in the fields since he was little?'

\end{itemize}
\z

In line 5, CH states that his father's wife, i.e. his mother, passed away when she was seventy. He refers to her using the overt form. In the next line, line 6, VA intervenes to ask whether his mother had passed away before his father, but refers to her using the \isi{zero form}. In line 7, CH corrects VA and responds by saying \textit{primeru laabe} `first him', using the overt form to make clear that it was his father who passed away first, not his mother. In line 8, VA picks up on the use of the overt form and uses it again to check that he has understood correctly, saying \textit{ah laabe ma gutibe} `ah, he already died'. In line 9, VA confirms this, repeating \textit{primeru laabe} `first him', using again the overt form to refer to his father. The use of the overt form to refer to the father continues throughout the rest of the interaction.

One of the outcomes of VA's turn in line 6, then, is that the \isi{zero form} is assigned to refer to CH's mother and the overt form is assigned to refer to his father. Rather than using a full NP to disambiguate reference, VA relies on the contrast between the two \isi{third person} forms to create a contrast between the father and mother. It is not a coincidence that the overt form was chosen to refer to the father, as he is the more thematic figure and the center of this conversational episode. In contrast, the \isi{zero form} is used for the mother, the less thematic figure.

This contrast between the overt enclitic and the \isi{zero form} in \isi{third person} is similar to the proximate/obviative contrast in \ili{Algonquian} languages, in which proximate forms are used for the \isi{third person} most central to the discourse and the obviative forms for more peripheral third persons \citep{dahlstrom1991,dahlstrom2003,dahlstrom2014}.\footnote{See, in particular, \citet{dahlstrom2014} in which the author argues that the definitions of both proximate or obviative cannot be reduced to that of \isi{topic} or \isi{focus}.} As with the proximate/obviative opposition, it would be interesting in future work to explore the extent to which the overt/zero alternation in ZAI can be sensitive to other factors such as empathy, agency, and point of view.


\section{Summary and conclusions}
This chapter summarized the \isi{pragmatic status} of the two types of \isi{third person} pronominal forms, the zero and the overt subject enclitic form, and explored the distribution and alternation of these forms in narrative and conversation. In addition to showing the syntactic facts governing the distribution of the overt and zero forms, this section showed that an important factor governing their use is the relative thematic \isi{salience} of the referents, wherein the overt pronoun is used for more thematic figures and the zero for less thematic figures. Again, the ZAI data is unusual in this regard as one would expect the reverse: highly topical participants to be zero-coded and less topical participants to be coded with overt forms.

\chapref{topicchapter} takes the analysis made in this chapter as a basis to consider the relationship between cognitive status and topichood and the expression of \isi{topic} relations between discourse referents and propositions. As will be seen, while cognitive status is not a prerequisite for topichood, \isi{topic} referents usually have a certain degree of pragmatic \isi{accessibility} such that more acceptable topics are higher on a cognitive status scale. First, I turn to an analysis of \isi{focus structure} in ZAI, which is the subject of the next chapter.

\lehead{\headmark}