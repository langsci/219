\addchap{Acknowledgments}
\begin{refsection}

This book is a revised version of my doctoral dissertation at the University of Chicago, which I successfully defended in April 2016. As with all language documentation projects, at its core this work is a collaborative endeavor and the product of a web of valuable relationships. My deepest thanks go to the community of \isi{Juchitán}, Oaxaca, and to the consultants and collaborators of this study, most especially, to my teacher Tom\'{a}s Villalobos Aquino. \textit{Diuxquixepe’ lii. Cadi tutiisi rini’ diidxaz\'{a}}. 

It is impossible for me to name everyone who made my own as well as my family’s fieldwork experience in \isi{Juchitán} such a rich and rewarding one. Thanks and appreciation cannot capture everything that I have learned and gained from the many relationships that have been built between many of the families in \isi{Juchitán} and mine. I am forever indebted to Tom\'{a}s Villalobos Aquino and Rosa L\'{o}pez V\'{a}squez and family for their unconditional support for me and my family from the first time we visited \isi{Juchitán}. They hosted, taught, and shared with us with such generosity and openness that I only hope I one day have the words to convey to others. There is nothing more meaningful that I can take away from this project than the friendship and care that they have shown me.

I would like to highlight the contribution of Irvin Villalobos L\'{o}pez, who worked conscientiously and consistently in reviewing, transcribing and translating many recordings. I owe a debt of gratitude to Miguel Villalobos Aquino and family. Their warm hospitality and enthusiasm for all things were infectious and a wonderful source of comfort, especially during so many months of intense heat. I cannot thank the Saynes V\'{a}squez family enough for all of the openness and generosity they showed me and my family. I am very grateful to Na Ernestina, Ta Cecilio, and Ta Ulises and family for all of their great help and friendship, and for offering a place to make hammocks and to just be. Na Maria Reina and family shared their home and their lives with us in the most generous and happiest of ways. I wish to thank Porfirio Matus Santiago for sharing so much of his language with me and with so much enthusiasm. I am thankful to everyone at Lidxi Guendabiaani’ for always welcoming me with open doors, literally. Yolanda L\'{o}pez G\'{o}mez offered a great deal of trust and unwavering support. 

%I would like to extend a sincere and warm thank you to everyone who participated in the orthography workshop in November 2011 at Lidxi Guendabiaani’, in particular Vicente Marcial Cerqueda, who included me with open arms. I learned a great deal and was inspired by the work of Natalia Toledo and Victor Cata during their first workshop on Zapotec literacy and creative expression entitled El Camino de la Iguana in November 2011. In December 2011, I was fortunate to participate over two sessions in the community of Santa Mar\'{i}a Guienagati with community members to develop an orthography for their language. I am very grateful to all of the teachers and staff of the Jos\'{e} F. G\'{o}mez Elementary School for hosting me at their campus for an orthography workshop for teachers in October 2012. I also am indebted to all of the teachers and staff of Heliodoro Charis Castro Elementary School for their insight and participation in an orthography workshop for teachers there from October-December 2012.

I wish to thank my editor Philippa Cook and for her dedication to this series and for her strong support of this project. I am grateful to the anonymous reviewers for their close reading and insightful critique and to the anonymous proofreaders for their attention to detail. Finally, I wish to thank Sebastian Nordhoff for his precise and energetic work in getting this book to press. All remaining errors are my own.

This work was stimulated and guided by my professors at the University of Chicago. I am extremely grateful to Lenore Grenoble for her tireless support of this project from the beginning and her sustained guidance has been astute, inspirational and energizing. Amy Dahlstrom was instrumental ever since I first began to analyze the overt/zero \isi{third person pronoun} alternation and the LA particle. Michael Silverstein provided much needed support, encouragement, and motivation through many enlightened and productive conversations.

%I am very grateful to all of my professors at the University of Chicago. In particular, I would like to thank Alan Yu for his confidence in helping me attempt to tackle Zapotec phonology for the first time. I am thankful to Salikoko Mufwene for many thoughtful discussions about the nature of language and the explanatory goals of Linguistics. Chris Kennedy has always given me tremendous support, especially in the early years in navigating graduate school. Outside of the Linguistics department, I feel extremely lucky for the opportunity to have learned from the mentorship and insight of John Lucy. I also wish to thank Barbara Pfeiler for her wonderful introductory course to the study of the indigenous languages of Mexico in Spring 2007. Before I came to the University of Chicago, I was introduced to the study of pragmatics by Carmen Curc\'{o}. I want to thank her for continuing to be a source of inspiration for me.

%This project would not have been possible without funding support from the University of Chicago. My first two trips to \isi{Juchitán} were made possible through two summer Foreign Language Area Studies (FLAS) grants for participation in the Isthmus Zapotec language program administered by San Diego State University. 

The main fieldwork for this project was made possible by grants from the Endangered Languages Development Programme (ELDP), the National Science Foundation's Documenting Endangered Languages program (NSF-DEL), and the Jacobs Research Fund. A large part of the writing stage was completed during a 12-month fellowship at the Smithsonian Institution. I am especially grateful to Gabriela P\'{e}rez B\'{a}ez for her generous and valuable help. I also would like to thank Mark Sicoli for his insightful comments and suggestions.

%My experience as a graduate student was nurtured in many positive ways through the support and camaraderie of the great collection of graduate students in the Linguistics department and, particularly, of my cohort, Thomas Grano, Yaron McNabb, Max Bane, April Grotberg, and Arum Kang. I have very much valued and appreciated the strong and fun mentorship of Jackie Bunting. I also would especially like to thank Les Beldo for his many years of thoughtful friendship that has now crossed so many life events.

I would like to especially thank Eduardo Toledo Garc\'{i}a who made not one journey but countless journeys possible between \isi{Juchitán} and Mexico City as well as across the Isthmus and within the state of Oaxaca, offering everything he could, including his car, his driving, his good humor, and his care. Patricia Bustamante Herrera and Donaj\'{i} Toledo Bustamante have treated me as a family member ever since I first met them. Hortencia Toledo and Vicente Orozco lovingly welcomed us into their home on so many occasions, I do not think I can remember them all. 

I am extremely thankful to my parents, Javier Bueno and Nina Holle, who have supported me and my family in countless ways. The same is true for my brothers, Carlos and Francisco, who have contributed in ways that they probably do not imagine. 

Finally, this work benefited greatly from the deep insight and solidarity of Nadxieli Toledo Bustamante. Paula has made the journey with us and has enriched every part of it. It is truly special to have shared our time in \isi{Juchitán} together and to now be able to share this with them.

\printbibliography[heading=subbibliography]
\end{refsection}

