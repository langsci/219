
\chapter{Conclusions and avenues for further research}


The fundamental aim of information structure studies, and of discourse pragmatics more generally, is to understand how the same propositional content can be expressed in linguistically different ways. In this, it is important examine the \textit{syntagmatic} relations between the elements of a clause or sentence and the ways that these can vary. More crucially, however, the study of information structure requires an analysis of the \textit{paradigmatic} relations between different, but related clause or sentence structures. These structures, as they are stored in the memory of speakers and hearers, represent alternative ways to structure propositions that differ depending on the pragmatic goals of the speaker. In other words, the study of information structure involves not only the relationships and orders between elements within a clause or sentence, but also the relationships between clauses or sentences that are semantically equivalent though formally and pragmatically different. These relationships are the paradigmatic relations that hold between available alternatives and that speakers and hearers bring to bear to accomplish their communicative goals. 

This study examined the paradigmatic relations that hold in ZAI between different structures on two distinct levels: a) the pragmatic states of the referents of individual sentence constituents in the minds of the speech participants, and b) the pragmatic relations established between these referents and propositions. First, as we saw in Chapter 3 and 4, speakers use the relationships between nominal forms, cognitive statuses, and grammatical roles in nuanced ways to accomplish specific communicative and interactional goals, such as to introduce and track referents, mark referents as more or less accessible, as well as to mark certain referents as more or less thematic. Second, as we saw in Chapters 5 and 6, speakers exploit the relations between constituent orders, morphology, and topical and focal material to distinguish between presuppositions and assertions, to mark shifts of background information or of topical units, and signal the focus domain of a proposition, as well as to accomplish interactional goals such as holding or ceding the floor in turn-taking in conversation.

With these two directions in mind, this chapter presents an overview of the main contributions of this study. In this, I discuss the conclusions derived from the analysis of the main information structure properties of ZAI, namely: 1) nominal forms and cognitive status, 2) the \textsc{la} particle, and 3) topic and focus constructions. This discussion includes the conclusions reached in the analysis of the use of each of these three properties in narrative and conversation including: the alternation between overt and zero third-person pronominal clitics, the use of the particle \textsc{la}, and the parallel, chiastic use of predicate focus and argument focus. Included in each section is a discussion of possible avenues for further research.


%To understand how information structure is encoded in ZAI it is necessary to understand the interrelationships between specific areas of ZAI grammar, namely nominal forms, constituent orders, discourse particles, and phonological patterns. This section summarizes the conclusions drawn in this study with respect to these areas.


\section{Nominal forms and cognitive status}

This study explored the relationship between form and distribution of nominals and between their form and function, analyzing the different forms that are used to introduce and track referents and to mark referents as more or less accessible. The discussion, framed between Preferred Argument Structure \citep{dubois2003} and the theory of Accessibility \citep{ariel2001}, showed that the fundamental mechanism driving the tendencies captured by PAS can be traced to the notion of accessibility. 

More specifically, the avoidance of new referents and lexical NPs in the A role was understood as an avoidance of referents in the A role with a low degree of accessibility. The tendency, in other words, is to \textit{avoid low accessible As.} The result is that highly accessible referents with less coding material are likely to occur in the A role. In contrast, low accessible referents with more coding material are unlikely to occur in that role and, instead, will more consistently occur in the O role. The S role exhibits a tendency in between the A and O roles in that it will often house previously mentioned, animate, salient, topical, and recent referents. At the same time, however, it will often function as a ``cognitive staging area" for the introduction of new referents at episode boundaries.

Moreover, because nominal forms indicate the status of their denotations as pragmatically more or less available in the speaker or hearer's mind, the forms of nominals that speakers use depends on the assumed cognitive status of the referents involved. That is, they depend on assumptions that a speaker can reasonably make regarding the addressee's knowledge and attention state in the specific context in which the form is used. Therefore, not only does type of nominal expression correlate with grammatical role, but with cognitive status as well. 

%This relation between grammatical role of core arguments, type of nominal expression, is summarized here:

%\singlespacing
%\ea\label{nomgramstat}Correlations between cognitive status, type of nominal expression, and grammatical role
%\begin{table}[H]
%\begin{center}
%%	\footnotesize
%%	\begin{adjustwidth}{-.5in}{-.5in}
%\begin{tabular}{| c | c  c  c  c  c  c |}\hline
% cognitive & \textsc{In} & \textsc{Activated} & \textsc{Familiar} & \textsc{Uniquely} & \textsc{Referential} & \textsc{Type} \\
%status  & \textsc{focus} &  & & \textsc{identifiable} &  & \textsc{identifiable} \\
%\hline
%type of & \textit{=b\v{e}}  & independent &  & NP + \textit{qu\v{e}} &  & \textit{ti} NP `a NP' \\
% nominal & \textit{=$\varnothing$}  &  pronoun &  & & & {$\varnothing$ N}   \\
% expression & & NP + \textsc{dist} & & & &  \\
%& &     & & &  &  \\
% \hline
% & & & & & & \\
%  grammatical & & A & $>$ & S  & $>$ & O  \\
%role & & & & & & \\
%& & & & &  & \\
%\hline
%\end{tabular}
%\end{center}
%\end{table}
%\z
%

It is important to note that pragmatic or cognitive status is not a pre-requisite for topic or focus-hood, although it may play a role. Because insufficiently accessible topic referents are more difficult to interpret, topic referents usually have a certain degree of pragmatic accessibility, where more acceptable topics are higher on a cognitive status scale (i.e., the Topic Accessibility Scale \citep{lambrecht1994}). The least acceptable are indefinite NPs and bare nouns. The most acceptable topics in ZAI are clitics. Related to this, it was observed that, although inconsistent, the inanimate object enclitic is employed relatively frequently for topics (cf. example (\ref{objectni})). One goal of future work should be to pay close attention to this use.

Correlations were also found between information structure of certain types of constructions and the cognitive status of the referents involved. \textsc{In focus} (Gundel et. al. (1993)) or \textsc{activated} referents do not occur in presentational or event-reporting constructions. \textsc{Type identifiable} referents do not occur in ``marked topic", detachment constructional involving the particle \textsc{la}. Therefore, for ZAI, NPs in presentational constructions are never pronominal forms, and NPs in detached, \textsc{la}-marked phrases are never indefinite. Presentational constructions are often used to introduce new, human referents, but new referents, either human or not human, can also be introduced in the O role using topic-comment constructions.

\sectref{alternation} focused on the pragmatic status of the two third person pronominal forms, the zero and the overt subject enclitic form, exploring the distribution and alternation of these forms in narrative and conversation. While the overt form was found to have a broader set of binding conditions than the zero form, the choice between the two forms is free at the main clause level. In those cases, an important discursive factor governing their use is the relative thematic salience of the referents. Because the overt pronoun is used for more thematic figures and the zero for less thematic figures, speakers must make active choices in contexts involving multiple third-person participants about which pronoun to assign to each. The study of narrative and conversational contexts is therefore crucial for understanding how speakers and hearers evaluate the relative thematicity of participants and use linguistic resources to do so.

%Can some accessibility markers (nominal forms) encode different degrees of accessibility through prosodic cues, e.g. separate intonation units?


\section{Topic and focus constructions}

At the center of information structure in ZAI is the flexible nature of constituent order. As we saw, the extent to which phonetic and intonational cues play a role in the expression of the cognitive status of referents was found to be minimal and information structure categories and relations are expressed mainly through manipulation of constituent order. 

Verb-initial clauses are compatible with the widest range of pragmatic construals as they can be employed in all topic-focus construction types: event-reporting, topic-comment, and identificational constructions. Constituent order, however, adapts to discourse functions and verb-initial syntax in ZAI is frequently violated in constructions in which topicalized and focalized elements may often appear before the verb. For this reason, we described ZAI as syntactically relatively flexible. In addition, because the focus domain is mostly tied to the pre-verbal position, ZAI can be described as relatively rigid pragmatically. Pre-verbal constituents, whether subjects, objects, or adjuncts, are almost exclusively focused constituents of identificational constructions.\footnote{One exception to this is the topicalization construction, in which the pre-verbal constituent is a subject-topic with a co-referring enclitic on the verb. These are used typically in cases of topic promotion.} 

Crucially, therefore, focus structure in ZAI may motivate certain syntactic arrangements. The reverse, that syntactic arrangements motivate changes in the focus domain, is never the case. 

Moreover, constituent order interacts closely with nominal form in the expression of topic and focus relations in ZAI. Lexical NPs in any of construction type typically signal a constituent that forms part of the focus domain. Independent pronominal forms, for their part, may signal topical or focal material, depending on position and on context. Meanwhile, dependent forms, i.e. subject enclitics, are used exclusively for subject-topics. A focused subject cannot appear as an enclitic on the verb. 

Finally, it was noted that both verb-initial and non-verb-initial structures exploit positions of prosodic prominence at the beginning and end of IUs. As we saw through an analysis of the use of different focus structure constructions in narrative and conversation, these positions are exploited in the parallel, chiastic use of predicate focus and argument focus. 

In this sense, while there is no evidence for pitch accents associated with topical or focal material, it is possible that there may be a prosodic motivation for the various types of constituent orders and the pragmatic motivations underlying their use. The search for description and explanation in this dimension would benefit greatly from a detailed, systematic study of the range of intonation patterns employed by ZAI speakers and their relation to the diversity of information structure categories and constructions. Ideally, this study could be extended or related to similar phenomena in related Zapotec languages.



\section{The \textsc{la} discourse particle}

The discourse particle \textsc{la} is involved in expressing information structure in ZAI. As we saw in Chapter 4, \textsc{la}-marked constructions can have a topic-promoting function, but also mark topical information, set the spatial, temporal, or individual framework within which the predication holds, and play a discourse cohesion role. They mark phrases that function as ``scene-setting topics," can have a frame-setting or delimiting function, can mark changes in topic or boundaries of topical units, and/or can function as contrastive topic markers.

More generally, constructions with \textsc{la} form part of the background presuppositions and establish a framework within which to proceed with the discourse, in much the same way that a question does. As was pointed out, there are, in fact, similarities between the use of \textsc{la} in yes/no questions and in \textsc{la}-marked or detached phrases in that both are used to secure referential common ground with the addressee(s). From this perspective, \textsc{la} functions as a `try'-marker and as a resource for negotiating common ground. 

As with the analysis of the overt versus zero alternation in third person pronominal forms, the multifunctional analysis of \textsc{la} also requires the analysis of spontaneous speech and, specifically, of conversation. It is likely that the use of \textsc{la} is tied to the ways that ZAI speakers signal degrees of awareness of common ground in interaction through not only linguistic means but through non-verbal means as well. An analysis of multi-modal interaction would no doubt be extremely worthwhile to begin to understand how forms such as this are employed and how they fit into local conversational norms about the kinds of assumptions that are made explicit linguistically between speakers and hearers and which are not. 

Because listeners in different speech communities can orient themselves in different ways, the following question is posed: How can the use of the particle be linked to local conversational strategies and norms? From this perspective, probably a characterization of \textsc{la}, as well as a more general characterization of the focal structure of the ZAI in terms of notions such as topic and focus is insufficient (see \citealt{matic2013,ozerov2015}). Instead, it is likely that the uses of the focal structure will be better understood through an analysis of the interaction; that is, through an analysis of the types of interactions that participants are having in the conversation and why.





%\section{Avenues of further research}

%Benefits of analyzing spontaneous speech, narrative and conversation in three areas: overt vs. zero, the \textsc{la} particle, and use of parallel chiastic structures. See various topic and focus constructions as a resource for floor-holding, turn-taking, and turn-entry points and in the co-construction of talk.

%Paradigmatic relations are exploited for interactional goals.

%Tied to the degree of awareness and the securing of common ground between participants and how it is signaled in conversation. 

%How do speakers and hearers orient to speech and conversation?

%Important for understanding the resources available for organizing talk and conversation and for understanding local conversational norms.

%Is the use of inanimate object enclitic tied to object-topic status? \ref{objectni}

%a) intonation, b) use and distribution of inanimate object enclitics, c) TAM system in discourse, d) causative-inchoative verbal alternation in discourse, e) local conversational strategies and norms

%phonology
%a detailed phonetic analysis of various utterance types, topic/focus, etc.
%a systematic (elicitation-style or experimental) study of the diversity of information structural types, including phonetic cues
%
%syntax
%pronominal objects that appear as enclitics must be topics? a change in course?
%
%conversation
%importance of holding the floor
%comparison of male vs. female speech 
%local conversational norms
