\chapter{Background: the basic grammatical structures of ZAI}\label{backgroundchapter}


This chapter presents a short description of the main typological characteristics of the language in which I summarize the aspects of ZAI grammar that are most relevant to the analysis of information structure. This description sets a foundation on which to explore the interrelationships between nominal forms, constituent orders, particles, and prosodic patterns. The chapter begins with a description of the segmental and tonal inventory and a brief explanation of the orthographic conventions used throughout. It then builds on an analysis of the ZAI tonal system to discuss the basic prosodic properties of the language at the phrase and discourse level, in particular the structural function of stress and pauses. The chapter then continues with a look at ZAI verbal forms and basic clause structure. This leads into an examination of the main constituent orders in ZAI and concludes with a closer inspection of the pre-verbal position. 



\section{The segmental and tonal inventory}\label{briefsketch}


In this section, I make a brief sketch of the segmental inventory and phonological system of ZAI. The information presented in this section is important to understanding the prosodic and verbal structures discussed in the remainder of the chapter.

 
\subsection{ZAI segmental inventory}

ZAI contains the segment inventory shown in Tables \ref{consonants} and \ref{vowels}.
\begin{table}[H]
\begin{center}
\begin{tabular}{| c | c | c | c | c | c |}\hline
p &  & t & t\textipa{S} & k & \\
% &  &  & (ch) &  (c/qu) &  \\
b &  & d & d\textipa{Z} & g &  \\
% &  &  & (dx) &  (g/gu) &  \\
 \hline
 & f* & s & \textipa{S} &  & h \\
% &  &  & (xh) &  & (j)  \\
 &  & z & \textipa{Z} &  &  \\
%  &  & z & (xh) &  &  \\
\hline
m &  & n & \textltailn &  &  \\
 &  & n: &  &  &  \\
 \hline
  &  & \textipa{r}* &  &  &  \\
% &  & (r/rr) &  &  &  \\
 &  & \textipa{R} &  &  &  \\
% &  & (r) &  &  &  \\
 \hline
 &  & l &  &  &  \\
  &  &  l: &  &  &  \\
\hline
w &  &  & y &  &  \\
%(hu) &  & (y) &  &  &  \\
\hline
\end{tabular}\\
\caption{\small{ZAI consonant inventory}}
\label{consonants}
(\small{* = Appear only in loanwords})
\end{center}
\end{table} 

The relevant contrast between consonants with the same place of articulation has traditionally been referred to as a fortis-lenis contrast (\citealt{pickett1960}, \citealt{pickett1998}; see also \citealt{arellanes2009}, \citealt{chavezpeon2010} with respect to other Zapotec languages).\footnote{This contrast has also been referred to as a morpho-phonological contrast between simple and geminate consonants \citep{swadesh1947}.} This fortis-lenis contrast parallels the voiced-voiceless distinction, where the lenis consonants are the voiced consonants and the fortis consonants are the voiceless consonants. 

In addition to five modal vowels, vowels may also appear glottalized or laryngealized (see \tabref{vowels}). 

\begin{table}[htp]
\begin{center}
\begin{tabular}{| c c c |  c c c |  c c c |}\hline
i & i\textipa{P} & i\super{\textipa{P}}i & & & & u & u\textipa{P} & u\super{\textipa{P}}u \\
\hline
e & e\textipa{P} & e\super{\textipa{P}}e & & & & o & o\textipa{P} & o\super{\textipa{P}}o \\
\hline
 & &  &  a & a\textipa{P} & a\super{\textipa{P}}a & & & \\
\hline
\end{tabular}\\
\caption{\small{ZAI vowel inventory}}
\small{(Modal, laryngealized, and glottalized vowels)}
\label{vowels}
\end{center}
\end{table} 
Glottalization is realized as a post-vocalic glottal stop in a stressed monosyllabic root (\ref{glottalized}a) (the prefix \textit{ri} is a habitual marker) and, if the root is disyllabic, also simultaneously as a word-final glottal stop in pre-pause position (\ref{glottalized}b). 

\ea\label{glottalized}
\begin{itemize}
\item[a.] \textit{ri-nda'} {[}r\`{i}`nd\`{a}\textipa{P}{]} `stinks' (cf. \textit{ri-nd\v{a}} {[}r\`{i}`nd\v{a}{]} `arrive')\\
\item[b.] \textit{b\'{e}'\~{n}e'} {[}b\'{e}\textipa{P}\textltailn\`{e}\textipa{P}{]} `alligator' (cf. \textit{be\~{n}e} {[}b\`{e}\textltailn\`{e}{]} `mud')\\
\end{itemize}
\z
Laryngealization is realized as creaky vowel quality and a double pulse to the syllable (\ref{larynge}a,b). 
\ea\label{larynge}
%\begin{tabular}{l l l}
\begin{itemize}
\item[a.] \textit{saa} {[}s\`{a}\super{\textipa{P}}a{]}  `music'\\
\item[b.] \textit{na-dx\v{i}ib\v{i}} {[}n\`{a}-d\textipa{Z}\v{i}\super{\textipa{P}}ib\v{i}{]} `fearful'\\
\end{itemize}
%\end{tabular}
\z
Glottalization and laryngealization each interact closely with stress in ways that are discussed in more detail below in \sectref{tones}.


\subsection{The tonal system}\label{tonalsystem}

There are three phonemic tones: high (H), rising (LH), and low (L). These tones, as they appear on monosyllabic and disyllabic morphemes, are shown in \tabref{surfacetones}.\footnote{One additional attested tonal pattern not shown here, LH L, is found only in loanwords, e.g. \textit{m\v{a}le} `compadre', \textit{\v{o}ra} `hour'.}

\begin{table}[htp]
\begin{center}
\begin{tabular}{| l | c | c |}\hline
 & Monosyllabic & Disyllabic \\
\hline
H & \textit{dx\'{e}} & \textit{l\'{e}xu} \\
& {[}d\textipa{Z}\'{e}{]} & {[}l\'{e}:x\'{u}{]} \\
 & `boy' & `rabbit' \\
\hline
LH & \textit{dx\v{i}} & \textit{y\v{u}z\v{e}} \\
& {[}d\textipa{Z}\v{i}{]} & {[}y\v{u}:z\v{e}{]} \\
 & `quiet' & `livestock' \\
\hline
L & \textit{ru} & \textit{benda} \\
& {[}r\`{u}:{]} & {[}b\`{e}n:d\`{a}:{]} \\
 & `cough' & `fish' \\
\hline
\end{tabular}\\
\caption{\small {ZAI tonal inventory on monosyllabic and disyllabic morphemes}}
\label{surfacetones}
\end{center}
\end{table}
Importantly, morphemes which contain a rising (LH) tone on the final syllable carry a floating H tone. The floating H tone appears on the final syllable of these words in isolation, but floats onto the following syllable utterance-medially. Two examples of words uttered in isolation are given in \tabref{floatingtones}, along with an example of these used in a phrase in which the first word now appears utterance-medially.

\begin{table}
\begin{center}
\begin{tabular}{| c | c |}\hline
Monosyllabic & Disyllabic \\
\hline
 \textit{n\v{e}} & \textit{dub\v{a}} \\
{[}n\v{e}:{]} & {[}d\`{u}:b\v{a}:{]} \\
 L\fbox{H} & L  L\fbox{H} \\
 `and' & `maguey' \\
\hline
\end{tabular} \\
\caption{\small{Morphemes with floating H tone}}
\label{floatingtones}
\end{center}
\end{table}

\begin{table}
\begin{center}
\begin{tabular}{| c |}\hline
Used utterance-medially \\
\hline
\textit{n{e}}  \textit{d\'{u}b\v{a}} \\
%L  H L\fbox{H} \\
`and maguey' \\
\hline
\end{tabular} \\
\end{center}
\end{table}

Whereas the word \textit{n\v{e}} is pronounced with a H tone in isolation, when used utterance-medially, the floating H tone appears on a following L tone syllable causing the word \textit{dub\v{a}} to be pronounced \textit{d\'{u}b\v{a}}.

Finally, it is important note that the various surface tone types are not all manifested with equal regularity. Pickett's \textit{Vocabulario} reports a frequency of 6\% for words that contain a syllable with a high (H) tone, 22\% for words that contain a rising (LH) tone, and 17\% that contain a floating H tone. Words containing only low (L) tone syllables are the most common, comprising about 55\% of the lexical inventory. In the next section, I explore the place of the ZAI tonal system within the broader prosodic system of the language.


\section{The structural function of prosody in ZAI}\label{prosody}

This section is concerned with the structural function of prosody in ZAI, that is, with the role of prosody in the segmentation of the speech signal into groups of words. In what follows, I first present a more detailed account of the ZAI phonological system than what was given above in \sectref{briefsketch} by offering a summary of the interrelationships between tone, laryngealization, glottalization, and stress. After a short review of the existing literature on the structural function of prosody in other Zapotec languages, I then explore some of the ways that tone, laryngealization, glottalization, and stress interact within the ZAI prosodic system. Finally, I touch briefly on the role of prosody in the marking of information structure, a discussion that will be taken up again in more detail in \sectref{focuschapter}.


\subsection{Tones, VQMs and stress}\label{tones}

Morphemes in ZAI may be either monosyllabic or disyllabic. As was shown above, ZAI has three phonemic tones: high (H), rising (LH), and low (L) and two voice quality modifications (VQMs), laryngealization and glottalization, that may participate in lexical contrasts. 

In addition to these, stress, although not lexically contrastive, also plays a key role in ZAI phonology. As a rule, there is only one stressed, double-moraic segment within each phonological word. In disyllabic words, stress falls on the initial syllable. Stressed syllables generally contain long vowels. There are two cases, however, in which the characteristically long, stressed vowel does not occur: 1) if the post-tonic syllable begins with a voiceless obstruent, a nasal, a liquid or a glide which undergoes gemination (geminates are not contrastive in ZAI), as in the di-syllabic words \textit{m\v{i}l\v{i}} [m\v{i}l:\v{i}:] `mullet' and \textit{chup\v{a}} [chup:\v{a}:] `two'; or 2) if the morpheme is glottalized, as in the disyllabic word \textit{b\'{e}'\~{n}e'} [b\'{e}\textipa{P}\~{n}e\textipa{P}] `alligator', in which case stress is heard only as heightened intensity and raised pitch register. In short, when stressed, the ZAI syllable nucleus may either be a long vowel (V:), a vowel plus a lengthened consonant (VC:), a laryngealized vowel (VV), or a glottalized vowel (V'). Clitics do not bear stress and maintain a CV structure.

\tabref{summary} summarizes the interactions between tones, laryngealization, glottalization, and stress in stressed monosyllabic and disyllabic morphemes (for words uttered in isolation).

\begin{table}[H]
\begin{center}
\begin{tabular}{| c | c  c | c  c | c  c |}\hline
 & \multicolumn{2}{c|}{plain} & \multicolumn{2}{c|}{glottalized} & \multicolumn{2}{c|}{laryngealized} \\
\hline
H & \textit{dx\'{e}:} &  \textit{\underline{l\'{e}:}xu:}  & \textit{ri-nd\'{a}'} & \textit{na-\underline{ya}n\'{a}'} &  &  \\
tone   & H & H L & L H & L L H &  &  \\
& `boy' & `rabbit' & `gets hot' & `hot/spicy' &  &  \\
  &  &  &  &  &  &  \\  
&  &  &  & \textit{na-\underline{ya'}n\'{i}'} &  &  \\
&  &  &  & L L H  &  &  \\
 &  &  &  & `clear' &  &  \\
\hline

LH & \textit{dx\v{i}:} & \textit{\underline{y\v{u}:}z\v{e}:}  & \textit{ri-nd\v{a}'} &  & \textit{n\v{u}u} & \textit{na\underline{dx\v{i}i}b\v{i}:} \\
tone & LH & LH LH  & L LH &  & LH  & L LH LH \\
& `quiet' & `livestock' & `gets bitter' &  & `there is' & `fearful'  \\
\hline
L\fbox{H} &  \textit{n\v{e}:} & \textit{\underline{du:}b\v{a}:} &  &  & \textit{b\v{u}u} & \textit{ri\underline{dxii}ch\v{i}:} \\
tone &L\fbox{H}  & L L\fbox{H} &  &  & L\fbox{H} & L L L\fbox{H}  \\
& `and' & `maguey' &  &  & `charcoal' & `be angry' \\
\hline
L & \textit{ru:} &  \textit{\underline{ben:}da:} & \textit{ri-nda'} & \textit{na-\underline{ya'}qui'}  & \textit{chii} & \textit{na\underline{dxii}bi'} \\
tone  & L & L L & L L & L L L  & L & L L L \\  
  & `cough' & `fish' & `stinks' & `burnt' & `ten' & `smooth' \\
\hline
\end{tabular}\\
\caption{\small{Tone, laryngealization and glottalization (in words uttered in isolation) (\underline{underline} notes the stressed syllable in disyllabic roots).}}
\label{summary}
\end{center}
\end{table}


If a morpheme is stressed, stress falls on the initial syllable. Duration is the primary phonetic indicator of stress as the stressed syllable must be heavy: either the vocalic nucleus is long or the post-tonic consonant is fortis (a geminate) leaving the vocalic nucleus short. Pre-pause syllables are also long.

However, three additional observations are important to note. First, when we compare morphemes in stressed and unstressed contexts, we see that the shortened syllables in unstressed and utterance-medial positions carry fewer tones. In particular, LH contour tones only arise on long syllables, i.e. on syllables that are either stressed or before a pause. When unstressed, the syllable nucleus is only a single vowel and the contour tones are `simplified' to a level H tone. This strongly suggests that the mora is the tone-bearing unit (TBU) and that the most appropriate representation is most likely one in which contours are composed of a sequence of level H and L tones linked to the mora. Second, the data also suggest that the L tone is the more unmarked of the two tones. In addition to being the most distributionally unrestricted tone, L is also always the one that is deleted in contour tone `simplification'.\footnote{Stress and tone have been argued to be closely interrelated in a number of languages (for general discussion, see \citealt{yip2002}; \citealt{zhang2002}). In particular, pitch movement has been shown to be more common under stress (\citealt{zhang2002};  \citealt{zoll2003}). This is true in ZAI as well as contour tones are shown to commonly surface on stressed syllables. An additional manifestation of this is that stressed L tones have a phonetically falling pitch whereas unstressed syllables with L tone are phonetically level tones.} 

Furthermore, this raises an important question about the relationship between the realization of contour tone and the structuring function of prosody in ZAI discourse: if contour tones in ZAI only occur on stressed syllables and before a pause, what is the distribution of stress and of pauses at the phrase- or discourse-level? Are they predictable? These questions are addressed in the following sections. First, I briefly review previous studies on Zapotec prosody.


\subsection{Previous studies on Zapotec prosody}

To my knowledge, the only extensive study that has been done on phrase-level prosody in a Zapotecan language has been the work of Mark Sicoli (2007; 2010). In his PhD study, \textit{A linguistic ethnography of tone and voice in a Zapotec region}, Sicoli devotes two chapters to an analysis of prosody in Lachix\'{i}o Zapotec (Eastern Zapotec) at both the word level and the phrase level. Although Lachix\'{i}o Zapotec and ZAI are only distantly related, it is not surprising that many of Sicoli's observations with respect to prosodic structure hold for ZAI as well. 

He describes Lachix\'{i}o Zapotec as a ``stress-timed" language where there is only primary (no secondary) stress which is non-iterative, that is, has at most one stress foot. In addition, Sicoli notes that emphasis is marked by a geminate medial consonant or stressed vowel of the primary stress foot and that this can serve focus functions by marking the edge of a phrase. 

Based on these observations, Sicoli goes on to analyze the intonational system as composed of four nested levels: the phonological word, the metrical foot, the intermediate phrase, and the intonation phrase. The maximal phonological word is composed of a clitic phrase with the following structure: [[proclitic [stressed root]] enclitic]. The metrical foot, the unit counted for rhythm, is trochaic. The intermediate phrase, a unit between the intonation phrase and the phonological word, is defined by phonetic cues such as phrase-final, non-phonemic lengthening. The intonation phrase is defined prosodically by the structure of boundary tones (phrase-final intonation patterns) and by optional cues, such as pause, breath, and non-phonemic lengthening of phrase-final vowels. 

Aside from boundary tones such as a L boundary tone that marks the ends of speakers' turns and a H boundary tone that indicates non-finality, two factors show that phonological phrasing can have morphosyntactic functions in Zapotec speech: 1) case is unmarked morphologically; and 2) body part nouns may combine with other nouns to form locational expressions (Sicoli 2007: 132).

Sicoli provides an illustrative example of the second of these. In Lachix\'{i}o Zapotec intermediate phrases help to distinguish between NPs that are grouped together as phonological phrases and those that form separate phonological phrases; this is most clearly seen in the use of body part nouns in ``quasi-prepositional" phrases (2007: 133).\footnote{For more work on body part nouns in Zapotec see e.g. \citet{maclaury1989}; \citet{lillehaugen2006}.} For example, the two-noun phrase \textit{lattsa n\'{i}kko} (lit. chest + dog) can be either a possessive construction meaning `the chest of a dog' or a locational construction meaning `the side of a dog' (2007: 134). In the possessive structure, the H final intermediate phrase tone is placed at the end of the first word (the possessed object), grouping these words as two phonological phrases [[lattsa:][nikko]]. For the locational reading, the second word receives a H final phrase tone that groups these words as a single phonological phrase [lattsa n\'{i}kko], thus indicating a prepositional use.\footnote{Sicoli also takes this as evidence for the existence of intermediate phrase tones as opposed to intonational pitch accents since they occur at the end of the phrase on an unstressed syllable. Mock (1988: 204), in her analysis of ZAI phonology, in fact uses a similar example as evidence that ``words in ZAI need not receive stress since stress ultimately occurs for discourse-related reasons." She does not, however, elaborate on this point.} Compensatory lengthening provides another phonetic cue. 


\subsection{Prosodic properties of intonation units in ZAI}\label{stressrule}

Otomanguean languages have long engaged researchers in the study of the phonetic realization and phonological complexity of stress, tone and vowel phonation (\citealt{arellanes2009}; \citealt{avelino2004}; \citealt{chavezpeon2010}; \citealt{mock1988}; \textit{inter alia}). With the objective of understanding in detail the interaction between stress, tone and vowel phonation at the word or root level, the main sources of data for these studies have been words and phrases elicited in isolation. This section complements this growing body of work by presenting a preliminary analysis of the sound patterns in intonation units in ZAI, using naturally-occurring data as evidence.

To review, ZAI has conserved a CV(CV) structure at the root level. Vowels bear one of three tones - low (the most frequent), high, and rising - and have three phonation types - modal, glottalized and laryngealized. At the root and word level, stress is assigned predictably to the first syllable of the root. The vowel of the stressed syllable is short when the following consonant is fortis, and long when the following consonant is lenis. Various types of extrametrical units can attach to a root, including tense, aspect and mood prefixes as well as pronominal enclitics, yet, stress assignment remains dependent on the root structure. In discourse, however, stress and vowel phonation may undergo lenition under certain circumstances. It is this process and the resulting patterns that are the focus here.  

In this section, as in the remainder of the study, I use the ``intonation unit" (IU) \citep{chafe1994} as the basis for transcription and analysis. The reason for this is that IUs have been shown to operate as a fundamental unit of cognitive processing, social interaction, and other domains (\citealt{chafe1994}; \citealt{dubois1993}; \textit{inter alia}). To recognize boundaries between IUs, I follow Du Bois et al. (1992:100) in identifying five major perceptual and acoustic cues: (1) a coherent or unified intonation contour; (2) a resetting of the baseline pitch level at the beginning of the unit (pitch reset); (3) a pause between two units; (4) a sequence of accelerated syllables at the beginning of the unit (anacrusis); and (5) a prosodic lengthening of the syllables at the end of the unit.\footnote{It is important to note that the presence of any of these is neither a sufficient nor a necessary condition, as many may occur for reasons other than an IU boundary and some may be difficult to identify under certain conditions.} This last cue, IU-final lengthening, is especially relevant for ZAI: the delimitation of IUs in ZAI is aided by the fact that  glottalized and laryngealized vowels in IU-final position are immune to the lenition process.

Chafe (1994) distinguishes between three types of IUs: 1) substantive, 2) regulatory and 3) fragmentary. The analysis that follows will focus on the prosodic properties that can be observed in substantive IUs, that is, IUs that convey ideas about events, states, or referents that participate in the communication of propositional content. The data in my corpus shows that, in substantive IUs, stress -- whose main phonetic correlate I assume to be duration -- resides in the last root of each constituent in a clause and lenites in all other elements towards the left. 

Consider the brief sequence of substantive IUs in (\ref{IU1}). The first line shows the superficial phonetic representation and the second line shows the morpheme-by-morpheme underlying representation.


\ea\label{IU1} (20120526$\_$R$\_$TVA: 52.6s-56.8s)
\begin{itemize}
\item[01]
\glll rak\'{a} gid\'{a}\super{\textipa{P}}a nis:a lu:n\v{i:} \\
raka\textsuperscript{H} gui\textsuperscript{LH}-daa nisa lu=ni\textsuperscript{LH}	\\
then \textsc{pot}-empty water face=\textsc{3sg} \\
\glt `Then water is emptied in it,'



\item[02]
\glll gy\'{a}:ba t\textipa{S}upa t\textipa{S}\'{o}na \textipa{Z}\'{u}:ba lu:ni l\'{a}: \\
gui\textsuperscript{LH}-yaba chupa\textsuperscript{LH} chonna\textsuperscript{LH} xuba' lu=ni\textsuperscript{LH} la\textsuperscript{H}	\\
\textsc{pot}-fall two three corn face=\textsc{3sg} \textsc{la} \\
\glt `(when you) add a few kernels of corn are added to it,'

\end{itemize}
\z

Stress is realized in the first syllable of the last root of each main verb and each argument NP. In the first line, stress falls on the verb root \textit{-da\super{\textipa{P}}a} `to empty'. This is observed in the rearticulated vowel which is fully realized. Stress also falls on the first syllable of \textit{nis:a} `water', which contains a modal vowel that is short, followed by a lengthened fortis consonant. The body-part term \textit{lu} `face', as head of the locative phrase, also receives stress and the modal vowel is therefore long. In the second line, stress falls again on the first syllable of the verb root, \textit{-yaba} `fall', and on the first syllable of \textit{\textipa{Z}ub\'{a}\textipa{P}} `corn'. These two words also contain long modal vowels. 

Other words, such as connectives (e.g. \textit{rak\'{a}} `then' in line 1) and modifiers (e.g. \textit{t\textipa{S}up:\v{a}} \textit{t\textipa{S}on:\v{a}} `a few' (lit. `two, three') in line 2) are not stressed. Because stress does not fall on the modifiers, the fortis consonants following the modal vowels in \textit{t\textipa{S}up:\v{a}} and \textit{t\textipa{S}on:\v{a}} are not fully lengthened. This can be seen if we compare them to the fortis consonant in \textit{nis:a}, in line 1, which does receive stress and is thereby considerably longer (146ms for /s/ in \textit{nis:a} vs. 84ms for /p/ in \textit{t\textipa{S}up:\v{a}} and 75ms for /n/ in \textit{t\textipa{S}on:\v{a}}). Note also that the modal vowel of the unstressed pronominal clitic \textit{=ni} `3\textsc{sg}' is lengthened in IU-final position, 151ms in line 1, but is short otherwise, 59ms in line 2. Similarly, \textit{=ni}  carries an underlying rising tone with a floating H and is pronounced with a rising tone in line 1 when lengthened in IU-final position, but is pronounced with a low tone when short in line 2 (and the H tone floats to the following syllable).

What emerges from an analysis of IU sequences such as that in (\ref{IU1}), is that stress in ZAI is predictable at the word or root level and is likewise predictable within substantive IUs. The relevant generalization can be stated in terms of syntactic constituency: the last root of each VP or NP constituent receives stress and stress lenites in all other elements to the left. 


\subsection{Prosody in ZAI information structure: some initial remarks}\label{prosodyis}

In the previous sections, I have briefly described the phonology of ZAI including its tonal system, with high, rising and low contrastive tones. As was seen, this tonal system interacts in complex ways with vowel phonation and a fortis-lenis distinction in consonants. In addition, I observed that stress operates at the phrase level, concluding that the last root of each VP or NP constituent receives stress and that stress lenites in all other elements to the left. 

This basic understanding of the phonological system of ZAI will make it possible in \sectref{focuschapter} to investigate the contribution of prosody to information structure in ZAI. There, I will take up the question of whether topic and focus constituents have a constant prosodic realization and whether stresses and pauses are involved in the realization of topic and focus structures. Since one common strategy in languages to communicate the status of a referent as new or focused is via pitch accent, one goal in that chapter will be to determine whether this strategy is available in ZAI as well. We will see, however, that the extent to which phonetic and intonational cues play a role in the expression of information structure in ZAI is minimal and that information structural categories and relations are expressed mainly through the manipulation of constituent order.

In the next section, I move on to a review of verb and clause structure and of constituent order correlations in ZAI. This will complete the brief description of the typological characteristics of the language that will set the foundation for the analysis in the remainder of the study.


\section{Clause structure and constituent order correlations in ZAI}\label{wordorder}


This section begins with a review of basic verbal morphology. It then addresses the question of constituent order correlations in ZAI to determine whether the language exhibits tendencies that correlate with V-O order rather than with O-V order, as has been claimed for most, if not all, Zapotec languages. I conclude the section, and the chapter, by examining the role that constituent order may play in the expression of information structure and present data that identifies the pre-verbal position as the locus for a variety of discourse functions, including the expression of topic and focus relations. 



\subsection{Verbal morphology}\label{verbalmorphology}

Like most verb-initial languages, ZAI employs verbal prefixes. Verbs obligatorily inflect for tense-aspect-mood (TAM). In addition to TAM, verbs also inflect optionally for causative (prefix).\footnote{Overall there is a tendency for suffixes to be associated with OV languages and prefixes with VO languages. However, this is only a unidirectional correlation: if all affixes in the language are suffixes, the language is more likely to be OV. This correlation is not a strong one, and prefixes in OV languages are not at all rare. In other words, we can say that OV languages more commonly have suffixes, but we cannot say that VO languages more commonly have prefixes \citep{dryer2007}.} Also, if the subject is not a full NP, the verb can be followed by a subject pronominal clitic. The basic order of the morphemes in the ZAI verb can be represented in this way:


\vspace{3mm}
\textsc{aspect}-\textsc{(causative)}-root-\textsc{(modifier)}=\textsc{(subject clitic)}
\vspace{3mm}


Verb roots may belong to one of four verb classes based on the aspectual prefixes they can combine with. Detailed studies of the morphophonemics of ZAI verb classes are provided in \citet{marlett1987}, \citet{enriquez2008}, and \citet{perez2015}.\footnote{For other foundational work on Zapotecan verb classes, see \citet{smithstark2002} and \citet{campbell2011}.}

A few additional comments are in order with respect to the TAM prefix.\footnote{\citet{pickett1998} describes the ZAI TAM system as essentially an aspectual system with only one tense prefix (future). \citet{mock1990}, describes the system as just aspect and mood, while \citet{suarez1983} describes the system as one that combines tense, aspect and mood. A complete study of the ZAI TAM system would be extremely valuable (see P\'{e}rez B\'{a}ez (2015); also Sicoli (2015) for the TAM system of Lachix\'{i}o Zapotec.} \tabref{TAMsystem} provides a list of the eight aspectual prefixes found in ZAI as well as a short summary of some of the observations made by previous scholars.  

\begin{table}[htp]
\begin{center}
\begin{tabularx}{\textwidth}{l l l}\hline
 Prefix & TAM & Description/Example \\
 \hline
\textit{ri-}, \textit{ru-} & Habitual & Used for habitual or repeated actions \\
 & & may be in past or present, never in future \\
 \hline
\textit{bi-}, \textit{gu-} & Completive & For finished actions, typically in past \\
 & & but not necessarily (e.g. future perfect) \\
 \hline
\textit{ca-}, \textit{cu-} &  Progressive & For continuing actions \\
 & & may be in past, present or future \\
  & & but may be temporal when used for future \\
  \hline 
 \textit{za-}, \textit{zu-} & Future & For future actions not yet begun, certain \\
 \hline
 \textit{ni-}, \textit{nu-}, \textit{\~{n}-} & Irrealis & For something that is contrary to fact; \\
 & & for something that did not happen \\
 \hline
\textit{gui-}, \textit{gu-} & Potential & Future action \\
 & & in relation to the time indicated by the main verb \\
 & & or in relation to utterance time \\
 & & used for subordinate clauses \\
  & & also, `to want' or `to like to' (in the future) \\
 & & in some imperative constructions \\
 \hline
\textit{hua-} & Perfect & For past actions that have occurred more than once \\
 & & also used in the negative to show a time \\
  & & during which something has not happened \\
  \hline
\textit{na-} & Stative & Forms a stative verb \\
& & more limited distribution \\
 & & occurs with about half of the roots to  \\
  \hline
 \end{tabularx}
\caption{\small{ZAI Tense-Aspect-Mood system}}
\label{TAMsystem}
\end{center}
\end{table} 


For the purposes of this study, the TAM prefix will be referred to as an aspectual prefix, but no claim is being made as to the specific syntactic-semantic function of these prefixes and a complete analysis of the ZAI TAM system is outside the scope of this project.

Finally, it should also be noted that there is no morphological case marking on nouns and there is no agreement between the verb and any of its arguments. Some features of ZAI that are canonical of most verb-initial languages are: adjectives generally follow nouns, possessive constructions are possessor final, and the use of prepositions rather than postpositions. I address constituent order correlations further in the next section, where I analyze the position of the verb with respect to the direct object.



\subsection{Constituent order correlations}\label{odercorrelationssection}

Previous research on ZAI has claimed that the most common arrangement of constituents is verb followed by the subject then any objects (\citealt{pickett1960}; \citealt{pickett1998}).\footnote{The same is true for most if not all Zapotec languages (see e.g. \citet{lee2000} for San Lucas Quiavini Zapotec (Central); \citet{beamdeazcona2004} for Coatl\'{a}n-Loxicha Zapotec (Southern); \citet{sonnenschein2005} for San Bartolom\'{e} Zoogocho Zapotec (Northern); \citet{sicoli2007} for Lachix\'{i}o Zapotec (Eastern)).} Verb-initial languages are much less common than verb-final languages \citep{payne1995}. However, it is also generally understood that ``no languages are rigidly verb-initial in the same sense that some languages are rigidly verb-final." (E. Keenan, quoted in \citet[455]{payne1995}). These two facts make the study of constituent order and of verb-initial languages challenging as there are well-known problems with establishing the relevant criteria to determine the basic constituent order in a language. Salient among these are two particular difficulties: 1) the order of subject and verb and the order of object and verb are often easier to identify while the order of subject and object is often more difficult to identify; and 2) pronouns may exhibit constituent order properties that differ considerably from lexical noun phrases.

In determining these patterns for a language, should the relevant criterion be one of frequency, of distribution, or of pragmatics? In constituent order typology, frequency has been the primary criterion used \citep{dryer2007}. It can be argued that differences in frequency often provide a more reliable test than other tests (where the difference is large enough). However, differences in frequency might be an artifact of a particular set of texts, due to genre specific or speaker idiosyncracies, for example, and one might therefore find very different frequencies in a different set of texts. Also, frequency counts of some languages may not reveal one order as noticeably more frequent than the other. Additionally, it can also be argued that because it is not part of the grammar of the language, frequency should not be used widely as a criterion \citep{dryer2007}.

A criterion of distribution refers to whether the fact that one order, found to be in some way less restricted in its distribution, can be used as an argument that it is more basic than another, more restricted order. In a similar fashion, one order in a language may be considered pragmatically neutral and another to have some added pragmatic effect. However, it may not be obvious that one order adds any additional elements and, instead, the two orders may simply have a difference in meaning (e.g. OV order may be associated with indefinite objects and VO order with definite ones). 


In this section, I analyze the correlates of various grammatical elements with the relative order of verb and object in order to determine a tendency in ZAI toward either verb-object (VO) order or object-verb (OV) order. As will be seen, all but two of the elements correlate with a VO order, as would be expected. The section that follows will discuss the subject position and will show that the exceptions to the V(S)O order are the ones that are pragmatically marked.

The universal tendencies associated with OV versus VO order are found in languages in which there is considerable flexibility of constituent order, even among languages in which one order outnumbers the other by a frequency of only 2 to 1 \citep{dryer2007}. These elements are listed in \tabref{ovvo}.

\singlespacing
\begin{table}[H]
\begin{center}
\caption{\small{Elements whose order correlates strongly with that of verb and object (Dryer 2007)}}
\begin{tabular}{| l | l |}\hline
   OV & VO  \\
\hline
 postpositions & prepositions \\
\hline
 adpositional phrase - verb & verb - adpositional phrase \\
\hline
 genitive - noun & noun - genitive \\
\hline
 manner adverb-verb & verb - manner adverb \\
\hline
 standard - marker & marker - standard \\
\hline
 standard - adjective & adjective - standard \\
\hline
 final adverbial subordinator & initial adverbial subordinator \\
\hline
 main verb - auxiliary verb & auxiliary verb - main verb \\
\hline
 predicate - copula & copula - predicate \\
\hline
 final question particle & initial question particle \\
\hline
 final complementizer & initial complementizer \\
\hline
 noun - article & article - noun \\
 \hline
 noun - plural marker & plural marker - noun \\
\hline
 subordinate clause - main clause & main clause - subordinate clause \\
\hline
 relative clause - noun & noun - relative clause \\
\hline
\end{tabular}\\
\label{ovvo}
\end{center}
\end{table}


Examples for each are provided in the following discussion.


\subsubsection{Use of prepositions}

ZAI uses prepositional phrases, as in the following two examples:

\ea\label{prep1}
\glll m\'{a} bietebe d\'{e} lu yaga qu\v{e} \\
ma'\textsuperscript{H} b.yete=be\textsuperscript{LH} de lu yaga que\textsuperscript{LH} \\
already \textsc{compl}-descend=\textsc{3.hum} \textsc{pp} face tree \textsc{dist} \\
\glt `He already came down from on the tree' 
\z

\ea\label{prep2}
\glll cuchabe c\'{a}ni nd\'{a}ani ti lari \\
c.u-cha=be\textsuperscript{LH} ca=ni\textsuperscript{LH} ndaani ti lari \\
\textsc{prog.caus}-fill=\textsc{3.hum} \textsc{pl=3.inan} \textsc{pp} one cloth \\
\glt `He (was) putting them in a shirt'		
\z

Prepositions in ZAI, if they are not borrowed from Spanish, are body part terms.\footnote{For more on the use of body-part terms in Zapotec languages, see e.g. \citet{maclaury1989} and \citet{perez2011b}.} In (\ref{prep1}), the body part term \textit{lu} `face' is used as part of the prepositional phrase \textit{de lu yaga que} `from on the tree' (lit. `from face tree that'). In this case, the prepositional phrase is headed by the preposition \textit{de} borrowed from Spanish. In (\ref{prep2}), the body part term \textit{ndaani} `stomach' functions as the prepositional head of the phrase \textit{ndaani ti lari} `in a shirt' (lit. `stomach one shirt').


\subsubsection{Adpositional phrase placed after the verb}

The examples in (\ref{prep1}) and (\ref{prep2}) above demonstrate that the position of adpositional phrases is after the verb, as expected for a language whose basic order is V-O.



\subsubsection{Genitive follows the possessed noun}

As would be expected in a language with VO order, lexical genitives follow possessed nouns in ZAI, as in (\ref{gen1}): 

\ea\label{gen1}
\glll cayaadxa ti dxumi p\v{e}ra badunguiiu \\
ca-yaadxa' ti dxumi\textsuperscript{LH} pe\textsuperscript{LH}ra badunguiiu \\
\textsc{prog}-be.missing one basket pear man \\
\glt `One of the man's baskets of pears was missing'
\z
In the complex subject NP, \textit{ti dxumi pera badunguiiu}, the lexical genitive \textit{badunguiiu} `man' appears after the possessed noun \textit{ti dxumi pera} `a basket of pears.'

In addition, possessive pronoun clitics also follow possessed nouns, as in (\ref{gen2}):

\ea\label{gen2}
\glll bidx\'{i}'babe l\'{u} xpicicl\'{e}tab\v{e} \\
bi-dxi'\textsuperscript{H} ba=be\textsuperscript{LH} lu x-bicicle\textsuperscript{H}ta=be\textsuperscript{LH} \\
\textsc{compl}-climb.up=\textsc{3.hum} face \textsc{poss}=bicycle=\textsc{3.hum} \\
\glt `He got on his bicycle'
\z
Here, the third-person subject clitic \textit{=be} appears as an enclitic on the possessed noun \textit{bicicleta} `bicycle', to which the possessive prefix \textit{x-} attaches.



\subsubsection{Manner adverbs follow the verb}

Manner adverbs may follow the verb, as in (\ref{manner1}), where the adverb \textit{nachaahui'} appears after the verb: 

\ea\label{manner1}
\glll biluxebe n\'{a}chaahui' \\
bi-luxe=be\textsuperscript{LH} na-chaahui' \\
\textsc{compl}-finish=\textsc{3.hum} \textsc{stat}-well \\
\glt `S/he finished well'
\z
They may also attach directly to the end of the verb root, as modifiers, as in (\ref{manner2}):

\ea\label{manner2}
\glll g\'{a}tachaahui ira gu\'{e}tabaadxi c\v{a} \\
g\textsuperscript{LH}-a'ta-chaahui' guira\textsuperscript{LH} guetabaadxi ca\textsuperscript{LH} \\
\textsc{imp}-lay-well all tamal \textsc{dem} \\
\glt `Lay down all the tamales carefully'
\z
Here, the verb root \textit{a'ta} `lay down' contains a glottalized vowel that is pronounced when stressed. In this case, the adverb \textit{chaahui'} appears immediately after the verb root and stress falls not on the verb root but on the adverb as it is the rightmost element of the verbal constituent. Stress lenites in all elements to the left, as we saw in \sectref{stressrule}. 

There are, however, cases in which an adverb may appear before the verb, as in (\ref{manner3}):

\ea\label{manner3}
\glll nachaahui b\'{i}luxeb\v{e} \\
na-chaahui' bi-luxe=be\textsuperscript{LH} \\
\textsc{stat}-well \textsc{compl}-finish=\textsc{3.hum} \\
\glt `S/he finished WELL'
\z
Cases such as this occur when information carried by the verb is presupposed and the manner adverb is asserted, or focused (cf. \ref{manner1}). These case are pragmatically-marked in the sense of \citet{payne1995}, as I will explore below in \sectref{pre-verbalpos}.

Variation in the relative position of main clause and adverbial clause is common in ZAI, as in many languages. Conditional clauses, for example, exhibit a universal tendency to precede the main clause \citep{haiman1978}. In this study, I consider this variation to be related to discourse pragmatics and to the communication of topical information. This will be explored in more detail in \sectref{topicchapter} where, the issue of subordinate adverbial clauses will be tied closely to the analysis of the \textsc{la} particle, which is the topic of \sectref{laparticle}.


\subsubsection{Order in comparative constructions is adjective-marker-standard}

The comparative construction currently used in ZAI, with order adjective-marker-standard, is a construction borrowed from the Spanish \textit{m\'{a}s que}. An example is shown in (\ref{comp}):

\ea\label{comp}
\glll jm\'{a} nahuinni j\~{n}aabe qu\'{e} bixhozeb\v{e} \\
jma\textsuperscript{H} na-huinni j\~{n}aa=be\textsuperscript{LH} que bixhoze=be\textsuperscript{LH} \\
more \textsc{stat}-small mother=\textsc{3.hum} than father=\textsc{3.hum} \\
\glt `His/her mother is younger than his/her father'
\z
The order here is adjective-marker-standard. The native ZAI comparative construction has not yet been documented. However, in San Lucas Quiavin\'{i} Zapotec, a central Zapotec language, the native comparative construction appears to also have an adjective-marker-standard order (Galant 2006), as in (\ref{compslq}):
\ea\label{compslq}
\gll Zyu\`{u}a'-ru' Lia Oli'eb loh Rrodriiegw \\
tall-\textsc{er} Ms. Olivia than Rodrigo \\
\glt `Olivia is taller than Rodrigo' \\
\z
It is likely that the native ZAI comparative construction would be similar.


\subsubsection{Initial adverbial subordinator}

ZAI has a long list of adverbial subordinators, all of which have been borrowed from Spanish: \textit{ora, lugar de, ante, dede, cada, para, cumu, modo, sinuque, sin}. All adverbial subordinators occur at the beginning of the subordinate clause. Some examples are:

\ea\label{ora}
\glll \v{o}r\'{a} c\'{a} l\'{a}, m\'{a} \'{a}ca licu\v{a}rn\v{i}  \\
o\textsuperscript{LH}ra ca\textsuperscript{LH} la\textsuperscript{H} ma'\textsuperscript{H} g\textsuperscript{LH}-aca licua\textsuperscript{LH}r=ni\textsuperscript{LH} \\
when \textsc{dem} \textsc{la} already \textsc{pot}-become blend=\textsc{3sg.inan} \\
\glt `At that time, blend it'
\z

\ea\label{ante}
\glll \v{a}nte de las \v{o}cho chuud\v{u}  \\
a\textsuperscript{LH}nte de las o\textsuperscript{LH}cho ch-uu=du\textsuperscript{LH} \\
before of the eight \textsc{pot}-go=\textsc{1pl.excl} \\
\glt `before eight we'll go'
\z

\ea\label{purti}
\glll p\v{u}rti m\'{a} las \v{o}cho de la ma\~{n}\v{a}na chuuzulu \\
pu\textsuperscript{LH}rti ma'\textsuperscript{H} las o\textsuperscript{LH}cho de la ma\~{n}a\textsuperscript{LH}na chuu-zulu=$\varnothing$ \\
because already the eight of the morning \textsc{pot}.go-begin=\textsc{3sg.inan} \\
\glt `because already at eight in the morning it was going to begin'
\z
As with the comparative construction, it is likewise unclear what the native clause-combining strategy is perhaps one of juxtaposition, but this is conjecture and requires further study.





\subsubsection{Auxiliary verb precedes main verb}
	
A minority of verbs can occur as an auxiliary verb. When they do, they appear before the main verb. One example is \textit{-anda} `be able to' in (\ref{auxmain}), followed by the main verb:

\ea\label{auxmain}
\glll {?`}zanda \'{i}g\'{a}nit\'{u} l\'{a}? \\
z-anda\textsuperscript{LH} gui\textsuperscript{LH}-gani\textsuperscript{LH}=tu\textsuperscript{LH} la\textsuperscript{H} \\
\textsc{fut}-be.able \textsc{pot}-be.silent=\textsc{2pl} \textsc{q} \\
\glt `Can you (all) be quiet?
\z


\subsubsection{Copula precedes the predicate} 

There is no copular construction in ZAI. However, nonverbal predicates occur at the beginning of the clause, as in the following example:

\ea
\glll mec\v{a}nico laab\v{e} \\
meca\textsuperscript{LH}nico laa=be\textsuperscript{LH} \\
mechanic \textsc{base}=\textsc{3.hum} \\
\glt `He is a mechanic'
\z



\subsubsection{Question particles}

Interrogative expressions in content questions in verb-initial languages most commonly occur at the beginning of sentences. This is true in ZAI as well. In the examples below, the question words \textit{panda} `how many' in (\ref{panda}) and \textit{pabia'} `how much' in (\ref{pabia}) occur sentence-initially:

\ea\label{panda}
\glll {?`}panda k\'{i}l\v{o}metru bixoo\~{n}elu raqu\v{e}?  \\
panda\textsuperscript{LH} kilo\textsuperscript{LH}metru bi-xoo\~{n}e'=lu' raque\textsuperscript{LH} \\
how.many kilometer \textsc{compl}-run=\textsc{2sg} then \\
\glt `How many kilometers did you run?'
\z

\ea\label{pabia}
\glll {?`}pabi\'{a} ruxoo\~{n}elu ira dx\'{i} ya? \\
pabia'\textsuperscript{H} ru-xoo\~{n}e-lu' guira\textsuperscript{LH} dxi ya \\
how.much \textsc{hab}-run=\textsc{2sg} all day \textsc{q} \\
\glt `How much do you run every day?'	
\z
Yes/no question particles in verb-initial languages most often also occur at the beginning of the sentence. In ZAI, however, such a particle is not obligatory and, in fact, is rarely used. The final particle \textsc{la} is required in yes/no questions:

\ea
\glll {?`}(\~{n}\'{e}e) biiyalu laabe l\'{a}? \\
\~{n}ee\textsuperscript{H} bi-uuya=lu' laa=be\textsuperscript{LH} la\textsuperscript{H} \\
\textsc{q} \textsc{compl}-see=\textsc{2sg} \textsc{base}=\textsc{3.hum} \textsc{la} \\
\glt `Did you see him/her?'
\z
The question \textit{{?`}\~{n}\'{e}e biiyalu laabe?}, without the \textsc{la} particle, would be ungrammatical. \footnote{One of the hypotheses examined in more detail in \sectref{topicchapter} is that the yes/no question particle \textsc{la} is related to the \textsc{la} particle involved in the marking of topical information.}


\subsubsection{Initial complementizer}

There is no overt complementizer in ZAI. An example is shown in (\ref{complementizer}):
\ea\label{complementizer}
\glll binadiaag\'{a} binda ti gaayu \\
bi-nadiaaga=a'\textsuperscript{H} bi-nda ti gaayu \\
\textsc{compl}-hear=\textsc{1sg} \textsc{compl}-sing one rooster \\
\glt `I heard a rooster sing'
\z


\subsubsection{Article appears before the noun}

It is common for the article to precede the noun in VO languages.\footnote{An additional, though weaker, correlation is that articles appear to be somewhat more common in VO languages than they are in OV languages.} There are no articles in ZAI. However, quantifiers such as \textit{ti} 'one' (\ref{ti}) and \textit{ca} \textsc{pl} (\ref{ca}) may precede the noun:
\ea\label{ti}
\glll ti badunguiiu \\
ti badunguiiu \\
one man \\
\glt `one/a man'

\z

\ea\label{ca}
\glll ca badunguiiu \\
ca badunguiiu \\
\textsc{pl} man \\
\glt `men'
\z
Both of these NPs are indefinite. To mark definiteness, ZAI employs demonstratives, which must appear after the verb:
\ea\label{ti2}
\glll ti badunguiiu qu\v{e} \\
ti badunguiiu que\textsuperscript{LH} \\
one man \textsc{dist} \\
\glt `that man'
\z

\ea\label{ca2}
\glll ca badunguiiu qu\v{e} \\
ca badunguiiu que\textsuperscript{LH} \\
\textsc{pl} man \textsc{dist} \\
\glt `those men'
\z
Unlike articles, the position of demonstratives does not exhibit a cross-linguistic correlation with respect to the order of object and verb. The use of demonstratives in discourse will be explored in more detail in \sectref{paschapter}.


\subsubsection{Plural marker - noun}

The plural marker \textit{ca} always precedes the noun in ZAI, as was shown above in (\ref{ca}). 


\subsubsection{Main clause - subordinate clause}

Many languages, including ZAI, exhibit considerable freedom in the position of subordinate clauses. In some cases, adverbial subordinate clauses in ZAI can precede the main clause, as was seen above in (\ref{ora})-(\ref{purti}). However, complement clauses follow the main clause, as shown here (cf. (\ref{want})-(\ref{say})): 


\ea\label{want}
\glll racaladxi Ju\'{a}n gu\'{e}ed\'{a} M\'{i}gu\'{e}l \'{i}x\'{i}' \\
ri=aca-ladxi Juan\textsuperscript{H} gu\textsuperscript{LH}=eeda\textsuperscript{LH} Miguel\textsuperscript{H}  guixi'\textsuperscript{H}  \\
\textsc{hab}=occur-gut Juan \textsc{pot}=come Miguel tomorrow \\
\glt `Juan wants Miguel to come tomorrow' 
\z

\ea\label{say}
\glll na Ju\'{a}n biiya Migu\'{e}l ca xcu\'{i}d\'{i} qu\v{e} \\
na Juan\textsuperscript{H} bi=uuya Miguel\textsuperscript{H} ca xcui\textsuperscript{H}di que\textsuperscript{LH} \\
say Juan Miguel \textsc{compl}=see \textsc{pl} child \textsc{dist} \\
\glt `Juan said Miguel saw the children' 
\z

	
	
\subsubsection{Noun - relative clause}

Almost all VO languages place the relative clause after the noun, as the following example illustrates. Here, the relative clause \textit{ni riree ndaani yuze} `that comes out of the stomach of the cow' follows the NP \textit{cuaju ca} `the rennet':

\ea
\glll cu\v{a}ju ca n\'{i} riree ndaani y\v{u}z\v{e} \\
cua\textsuperscript{LH}ju ca\textsuperscript{LH} ni ri-ree ndaani yu\textsuperscript{LH}ze\textsuperscript{LH} \\
rennet \textsc{dem} \textsc{rel} \textsc{hab}-leave stomach cow \\
\glt `The rennet that comes out of the stomach of the cow'
\z



\subsection{Summary of constituent order correlations}

The above discussion has shown that the great majority of the constituent order correlations in \tabref{ovvo} conform to a pattern of verb-object in ZAI. A summary of which of these correlations hold in ZAI and how they are manifested is presented in \tabref{ovvo2}:

\singlespacing
\begin{table}[H]
\begin{center}
\caption{\small{Correlations between verb and object order in ZAI}}
\begin{tabular}{| l | l |}\hline
VO order correlations &  ZAI \\
\hline
 prepositions & \checkmark  \\
 \hline
 verb - adpositional phrase & \checkmark \\
\hline
noun - genitive & \checkmark  \\
\hline
verb - manner adverb & Variable, obeys discourse motivations \\
\hline
marker - standard & \checkmark (*native construction unknown) \\
\hline
 adjective - standard & \checkmark (*native construction unknown) \\
\hline
 initial adverbial subordinator & Variable, obeys discourse motivations \\
\hline
auxiliary verb - main verb & \checkmark \\
\hline
 copula - predicate & No copula \\
\hline
initial question particle & Yes/no particle appears clause-finally\\
\hline
initial complementizer & \checkmark \\
\hline
article - noun & No articles \\
\hline
plural marker - noun & \checkmark \\
\hline
main clause - subordinate clause & Variable, obeys discourse motivations \\
\hline
noun - relative clause & \checkmark\\
\hline
\end{tabular}\\
\label{ovvo2}
\end{center}
\end{table}


While the majority of the constituent order correlations discussed conform to cross-linguistic tendencies for VO languages, it is worth noting the exceptions here. First, there is no copula or articles in ZAI. Second, the principal rigid exception is the yes/no question particle \textsc{la}, which appears utterance-finally rather than, as would be expected for an VO language, utterance-initially. This particle will be analyzed in more detail in \sectref{laparticle}. Finally, several constituent order correlations show variation. We saw that in the cases of the orders manner adverb - verb or main clause - subordinate clause, the order obeys specific discourse motivations. These motivations will be explored more fully in Chapters \ref{focuschapter} and \ref{topicchapter}. The next section follows up this discussion of constituent order by focusing more specifically on the pre-verbal position, which we know to be a prominent position cross-linguistically and, in particular, in verb-initial languages.


\subsection{The pre-verbal position and rigidity in verb-initial syntax}

In her analysis of the pragmatic properties of verb-initial languages, \citet{payne1995} surveys the discourse functions that constituents may have in pre-verbal position. She groups these functions under the label ``pragmatically marked", that is, ``information which is to some degree counter to what the speaker assumes are the hearer's current expectations or presuppositions" \citep[110]{payne1995}. Payne argues that there exists a continuum for pragmatically marked (PM) information which includes, on one end, information that is contrary to hearer's assumptions and, on the other, information in accord with or only incrementally different from the hearer's expectations. Based on this observation, Payne proposes a hierarchy of pragmatic markedness, represented in \tabref{pragmaticmarkednesstable}:

\singlespacing
\begin{table}[H]
%\begin{adjustwidth}{-.5in}{-.5in}
\begin{center}
\begin{tabular}{| c c c c c |}\hline
more marked & & $>$ & & less marked \\
 & & &  & \\
NP in descriptive or &  $>$ & NP establishing & $>$ & Pragmatically  \\
background clause &  & a foundation &  &  marked NPs \\
\hline
\end{tabular}\caption{\small{A hierarchy of pragmatic markedness (Payne 1995: 479)}}
\label{pragmaticmarkednesstable}
\end{center}
%\end{adjustwidth}
\end{table} 
According to this hierarchy, if a verb-initial language places phrases before the verb to accomplish any function to the left on the following hierarchy, all phrases that accomplish functions to the right on the hierarchy will also occur before the verb. That is, among PM phrases, if a verb-initial language places a somewhat more-marked phrase type before the verb, then it will also place less marked types before the verb. Languages that fall to the left on this hierarchy are clearly less rigidly verb-initial than are languages to the right.

As will become clear from the following discussion, however, ZAI is not a rigidly verb-initial language. Indeed, all of the elements in the hierarchy -- from descriptive and background clauses to pragmatically-marked NPs -- are eligible to appear in pre-verbal position. I discuss the pre-verbal position in more detail in the next section as this is an important fact and one that I will return to throughout the analysis in the remainder of this study. It will become especially relevant in \chapref{focuschapter} and \chapref{topicchapter} when I discuss the question of the relative ``rigidity'' of ZAI syntax and its relation to the types of topic and focus constructions available to ZAI speakers. 


\subsection{The pre-verbal position in ZAI}\label{pre-verbalpos}

In rigid verb-initial languages, predicates also come first in clauses that are not temporally sequenced but which serve to introduce and describe referents, state background conditions, or describe events that are out of sequence with the main event line \citep[454]{payne1995}. An almost universal strategy in verb-initial languages, however, is that if part of a sentence is questioned or is the answer to a question, it will come first. They are, in the words of \citet{payne1995}, ``pragmatically marked," in the sense that initial position is associated with novel attention re-direction of some kind. The remaining constituents come at the end. 

The pre-verbal position has been identified as a privileged position from the perspective of information structure in other Zapotec languages as well. For example, \citet{broadwell2002} for San Dionicio Ocotepec Zapotec (Central Zapotec) and \citet{lee2000} for San Lucas Quiavini Zapotec (Central Zapotec) also identify the pre-verbal position as a topic or focus position. Similarly, \citet[103]{black2000}, in her study of Quiegolani Zapotec (Central Zapotec) syntax, states, ``Discourse analyses done on other Zapotecan languages show that the fronted nominal may be either old or new information.'' 

In addition to much of the data already explored above involving constituents in pre-verbal position (cf. adverbial clauses (\ref{ora})-(\ref{purti})); also, adjectives, as in  (\ref{manner3})), the patterns described below provide further evidence that the pre-verbal position in ZAI is indeed the locus for a variety of discourse functions, such as: question words, negation, focus of contrast (e.g. subject or objects NPs, adjectives), and initiation of new subsections of a text through the (re)introduction of participants.

\subsubsection{Pre-verbal position: \textsc{wh}-words}
As seen above in (\ref{panda}) and (\ref{pabia}), the pre-verbal position is reserved for \textsc{wh}-words. Two additional examples are provided here in (\ref{wh}) and (\ref{wh2}): 

\ea\label{wh}
\glll {?`}xi b\'{i}'nib\v{e}? \\
xi\textsuperscript{LH} b-i'ni-be\textsuperscript{LH} \\
what \textsc{compl}-do-3\textsc{sg} \\
\glt `What did s/he do' 
\z

\ea\label{wh2}
\glll {?`}tu b\'{i}'ni n\v{i}? \\
tu\textsuperscript{LH} b-i'ni ni\textsuperscript{LH} \\
who \textsc{compl}-do 3.\textsc{inan} \\
\glt `Who did it'  
\z

\subsubsection{Pre-verbal position: negation}
Negation in ZAI always precedes the verb, as in (\ref{neg}):
\ea\label{neg}
\glll qu\'{e} reedab\'{e} gu\'{i}r\'{a} dx\'{i} \\
 que\textsuperscript{H} r-eeda\textsuperscript{LH}-be\textsuperscript{LH} guira'\textsuperscript{LH} dxi \\
\textsc{neg} \textsc{hab}-come-3\textsc{sg} all day \\
\glt `S/He doesn't come every day' \hfill (Pickett, et al. 1998:78)

\z

\subsubsection{Pre-verbal position: focus of contrast}
Pickett, et al. (1998) note that a core argument can be ``emphasized'' by placing it before the verb. In such constructions, if the argument is a full noun phrase, no co-referring subject clitic pronoun is found on the verb, as shown in (\ref{emphnoun}):

\ea\label{emphnoun}
\glll P\v{e}dro biiya ti badudxaapa \\		
Pe\textsuperscript{LH}dro bi-uuya ti badu-dxaapa \\
Pedro \textsc{compl}-see \textsc{indef} child-woman \\
\glt `PEDRO saw a girl' \hfill (Pickett, et al. 1998:98)

\z
If the argument is a pronominal subject, however, a co-referring dependent pronoun does appear cliticized to the verb, as shown here in (\ref{emphpronoun}):

\ea\label{emphpronoun}
\glll laabe b\'{i}'yabe t\'{i} badudxaapa \\		
laa-be\textsuperscript{LH} b-i'ya-be\textsuperscript{LH} ti badu-dxaapa \\
\textsc{base}-3\textsc{sg} \textsc{compl}-see-3\textsc{sg} \textsc{indef} child-woman \\
\glt `S/HE saw a girl' \hfill (Pickett, et al. 1998:98)
\z
Additionally, a construction which places the object in pre-verbal position is also possible in ZAI. For example, in answer to the question `What did s/he do?' (\ref{wh}), one can respond:

\ea\label{preverbalobj}
\glll dxii\~{n}a bi'nib\v{e} \\
dxii\~{n}a' bi-ini=be\textsuperscript{LH} \\
work \textsc{compl}-do=\textsc{3sg} \\
\glt `S/He did WORK'
\z 
It is possible, also, to use a similar construction involving the discourse particle, \textsc{nga}.  

\ea\label{preverbalnga}
\glll dxii\~{n}a ng\'{a} bi'nib\v{e} \\
dxii\~{n}a' ng\'{a} bi-ini=be\textsuperscript{LH} \\
work \textsc{nga} \textsc{compl}-do=\textsc{3sg} \\
\glt `S/He did WORK'
\z 
In this case, the relevant interpretation is that of an exhaustive listing. A more detailed discussion of this particle will be taken up in \sectref{ngaargfoc}.

Although it is not clear what Pickett, et al. refer to by ``emphasized'', it is clear that the use of an NP in pre-verbal position in each of these cases communicates discourse-pragmatic information. In Chapters \ref{focuschapter} and \ref{topicchapter}, I analyze these constructions as ``identificational'' or ``argument focus" constructions, where only a single NP is focused and the rest of the proposition is within the presupposition \citep[228-233]{lambrecht1994}. As will be shown, in these cases, the NP in pre-verbal position is not necessarily ``new" information as it is not the focused noun itself which contributes the new information to the discourse, but the relationship between (the referent of) this noun and the entire proposition.  


\subsubsection{Pre-verbal position: left-dislocated phrases}
Finally, as will be discussed in more depth in Chapters \ref{paschapter} and \ref{topicchapter}, there may be nouns (including independent pronouns) that appear in the pre-verbal position and which are separated by the particle \textsc{la} as well as by a pause in the intonation. These are left-dislocated phrases, i.e. phrases that occur under a separate intonation contour, and which may or may not be morphosyntactically related to the verbal case frame. If related, a resumptive reference may occur. These left-dislocated phrases often delimit a time, location, or some other conceptual frame of reference for what follows. By contrast, a non-dislocated pre-verbal phrase may or may not be related to the verbal case frame, but, if it is, a resumptive reference will likely not occur. 


\section{Summary and research questions}

In summary, this chapter has described the main phonological and syntactic characteristics at the core of the grammar of ZAI. It was shown that ZAI is a tonal language, with high, rising and low contrastive tones and that these interact in complex ways with vowel phonation and a fortis-lenis distinction in consonants. It was also shown that stress and tone play a significant role in prosody beyond the word-level. Verb morphology is primarily agglutinative, that there is no morphological case marking on nouns and that there is no agreement between the verb and any of its arguments. I then reviewed the main patterns in constituent order relations in ZAI and showed that the most common arrangement of constituents in ZAI is considered to be verb followed by subject then object. Finally, many features of ZAI are characteristic of verb-initial languages: adverbial subordinators are clause-initial; use of prepositions rather than postpositions; adjectives generally follow nouns; possessive constructions are possessor final, etc. However, verb-initial syntax is often violated as the pre-verbal position can be the locus for important discourse functions.

With this background in mind, I devote the chapters that follow to an examination of the interplay between verb-initial order, tone and prosody in ZAI. As has been pointed out, little has been said about the possible phonological, morphological and/or syntactic correlations with the expression of information structure in this language. From the preceding discussion, however, several questions arise that will guide the analysis with respect to four areas: 1) the relation between nominal forms and cognitive status; 2) constituent order; 3) discourse particles; and 4) prosody. I list these questions here: 

\singlespacing
\vspace{3mm}
\noindent \textit{Nominal forms and cognitive status}
\begin{itemize}
\item How do the different morphological forms of nominals express different cognitive statuses? How does each cognitive status correlate formally with type of nominal expression?
\item To what extent do phonetic and intonational cues play a role in the expression of cognitive status?
\end{itemize}

\vspace{3mm}
\noindent \textit{Constituent order}
\begin{itemize}
\item Verb-initial syntax in ZAI is frequently violated in constructions in which topicalized and focalized elements may often appear before the verb. Since constituent order is known to have important discourse functions in many languages and since a small percentage of the world's languages are verb-initial, how does verb-initial syntax in ZAI condition the ways that speakers mark topic and focus? 
\item Are constituent order changes a possible strategy for expressing all types of topic and focus constructions or only a subset? How pragmatically and syntactically ``rigid" is the language?
\end{itemize}

\vspace{3mm}
\noindent \textit{Discourse particles}
\begin{itemize}
\item There are two discourse particles, \textsc{la} and \textsc{nga}, that are involved in expressing information structure in ZAI. Can the \textsc{la} form be considered a contrastive topic marker? Is the \textsc{nga} form involved in the realization of focused material? 
\item In which cases is the use of these particles infelicitous?
\end{itemize}

\vspace{3mm}
\noindent \textit{Prosody}
\begin{itemize}
\item If the realization of contour tones is tied to the realization of stress and of pauses, what is the distribution of stress and of pauses at the phrase- or discourse-level? Are they predictable? 
\item Are stress and pauses involved in the realization of topic and focus structures? Do topic and focus structures have a constant prosodic realization? That is, is prosody involved in the realization of topic or focus?
\end{itemize}


In the next chapter, I take the grammatical information presented here as a basis to address the first group of questions listed above with respect to ZAI nominal and pronominal forms and their potential functions in discourse. In particular, I explore the ways in which different forms may signal different types of cognitive status, terms which will be illustrated below. 

